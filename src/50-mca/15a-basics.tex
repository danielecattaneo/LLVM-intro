% !TEX root = main.tex

\subsection{Basics of LLVM-MCA}


\begin{frame}{Sum Reduction}
\begin{block}{Algorithm}
Given a vector $V = [v_1, v_2, \ldots v_N]$ we compute:\\
\[
x = \sum_{i=1}^{N}{v_i}
\]
\end{block}
\begin{block}{Obvious Implementation}
\cinput{listings/01_add_reduction_v1_cleaned.c}
\end{block}
\begin{itemize}
\item Let's measure the performance we get with LLVM-MCA!
\end{itemize}
\end{frame}


\begin{frame}{Marking portions of code}
\begin{itemize}
\item We need to tell LLVM-MCA which portion of code we want to measure!
\end{itemize}
\bigskip
Important things to know:
\begin{enumerate}
\item branch instructions are ignored
\item every piece of code is assumed to be executed in a loop
\item the behavior of memory accesses is not simulated
\end{enumerate}
\bigskip
Consequences:
\begin{itemize}
\item We need to debug data locality issues \alert{first}
\item The only thing that can be analyzed feasibly is straight-line code
\end{itemize}
\end{frame}


\begin{frame}{Marking portions of code}
In this example there is not much possibility for data locality
issues (only sequential reads...)\\
\bigskip
From what we said it's clear that the portion of code we are interested
in is the \alert{body of the loop}\\
\medskip
To make LLVM-MCA analyze it we have to enclose it \alert{in the assembly code}
with:\\
\smallskip
\begin{center}
\large
\texttt{\# LLVM-MCA-BEGIN }\textit{block\_name}\\
\smallskip
\texttt{\# LLVM-MCA-END }\textit{block\_name}\\
\medskip
{\footnotesize \textit{block\_name} is any string identifier you want}
\end{center}
\end{frame}


\begin{frame}{Marking portions of code}
If we are working in the C language, we can use \texttt{asm} blocks:
\begin{block}{Obvious Implementation (annotated)}
\cinput[\tt\small]{listings/01_add_reduction_v1.c}
\end{block}
\bigskip
\alert{Warning:} Adding \texttt{asm} blocks can prevent optimizations!\\
\smallskip
{\footnotesize In this particular case, it prevents \alert{loop unrolling}.
For the purpose of the example it's not a problem because we will talk about
loop unrolling in a bit...}
\end{frame}


\begin{frame}{Marking portions of code}
Let's compile with \texttt{-O3}...
\begin{block}{Obvious Implementation (compiled)}
\asminput[\tt\small]{listings/01_add_reduction_v1.s}
\end{block}
\bigskip
\end{frame}


\begin{frame}{Running LLVM-MCA}
It's now time to actually run the analysis with LLVM-MCA.\\
\medskip
There are several kinds of options available:
\begin{itemize}
\item Machine type options (defaults autodetected from host)
\smallskip
\item Analysis configuration options
	{\small
	\begin{description}[\texttt{-bottleneck-analysis}]
	\item[\texttt{-iterations}] Number of iterations to simulate
	\item[\texttt{-register-file-size}] Size of the shadow register file
	\item[\texttt{-bottleneck-analysis}] Enables bottleneck analysis
	\end{description}
	\smallskip
	}
\item Options for toggling \alert{views} of the analysis data:
	{\small
	\begin{description}[\texttt{-instruction-tables}]
	\item[\texttt{-resource-pressure}] Enable the resource pressure view.
	\item[\texttt{-instruction-info}] Enable the instruction info view.
	\item[\texttt{-instruction-tables}] Prints resource pressure information based on the static information available from the processor model.
	\end{description}
	}
\end{itemize}
\end{frame}


\begin{frame}{Running LLVM-MCA}
We'll use the following options:
\begin{itemize}
\item \texttt{-timeline}
\item \texttt{-all-stats}
\item \texttt{-bottleneck-analysis}
\end{itemize}
\medskip
The command line syntax is very simple.
\begin{block}{LLVM-MCA invocation command}
\tt \$ llvm-mca add\_reduction.c \textbackslash \\
\hphantom{MMMM}-timeline -all-stats -bottleneck-analysis
\end{block}
\end{frame}


\begin{frame}{The Views of LLVM-MCA}
The output is partitioned for each \alert{code region} and for each \alert{data view}.\\
\smallskip
In our example there is just one region.\\
\medskip
The main view presents generic information about the simulation.
%
\begin{columns}[onlytextwidth]
%
\column{0.45\textwidth}
\begin{block}{The Main View}
\txtinput[\tt\small]{listings/01_add_reduction_v1_p01.txt}
\end{block}
%
\column{0.5\textwidth}
\begin{description}[ABC]
\item[\texttt{Iterations}:] Number of iterations of the loop that have been simulated.
\item[\texttt{Instructions}:] Total number of instructions simulated.
\item[\texttt{Total Cycles}:] Number of clock cycles taken for performing the code.
\item[\texttt{Total uOps}:] Number of CPU microcode instructions issued.
\end{description}
%
\end{columns}
%
\end{frame}


\begin{frame}{The Views of LLVM-MCA}
\begin{columns}[onlytextwidth]
%
\column{0.45\textwidth}
\begin{block}{The Main View}
\txtinput[\tt\small]{listings/01_add_reduction_v1_p01.txt}
\end{block}
%
\column{0.5\textwidth}
\begin{description}[ABC]
\item[\texttt{Dispatch Width}:] Maximum number of microinstructions that can be issued simultaneously
\item[\texttt{uOps Per Cycle}:] Average number of microinstructions executed per clock cycle.
\item[\texttt{IPC}:] Average number of instructions executed per clock cycle.
\item[\texttt{Block RThroughput}:] Average \alert{reciprocal throughput} of the code block
\end{description}
%
\end{columns}
%
\end{frame}


\begin{frame}{Reciprocal Throughput}
\begin{center}
The \alert{throughput} is the \emph{amount of work done in a given time}\\
\medskip
The \alert{reciprocal throughput} is \emph{how much time is required to perform the unit of work}\\
%
\bigskip
\begin{tabular}{>{\raggedleft}p{0.4\textwidth} c p{0.4\textwidth}}
\emph{unit of work} & = & one iteration of the loop\\
\emph{time} & = & clock cycles \\
\end{tabular}\\
\bigskip
%
The reciprocal throughput given by LLVM-MCA is \alert{theoretical}.\\
It does not take into account data dependencies amongst multiple iterations.\\
\medskip
It depends from the resource usage of the block of code.
\end{center}
\end{frame}


\begin{frame}{Reciprocal Throughput}
\begin{center}
The IPC (Instructions per Cycle) estimation \alert{does} take into account data dependencies.\\
\bigskip
Since IPC (like MIPS) is a measure of \alert{throughput}, we can derive the \alert{maximum throughput} by computing the \alert{inverse of the RThroughput}\\
\bigskip
\[
\textsf{IPC} \le \frac{N_{block}}{\textsf{Block RThroughput}}
\]
\[
N_{block} = \frac{\textsf{Instructions}}{\textsf{Iterations}}
\]
\end{center}
\end{frame}


\begin{frame}{Data dependencies are important!}
Let's try calculating the maximum IPC for our example...
\[
\textsf{IPC}_{max} = \frac{N_{block}}{\textsf{Block RThroughput}} = \frac{1}{0.5} = 2
\]
The maximum IPC is much higher than the \alert{actual} IPC ($0.25$)!\\
\medskip
This is because \alert{every instruction executed depends on the result of the previous one}
\end{frame}


\begin{frame}{The Timeline View}
We can verify visually if an instruction is waiting for another one using the \alert{timeline view}.
\bigskip
\begin{block}{Timeline View}
\txtinput[\tt\fontsize{6.7pt}{8pt}\selectfont]{listings/01_add_reduction_v1_p02.txt}
\end{block}
\end{frame}


\begin{frame}{The Timeline View}
\begin{block}{Timeline View}
\txtinput[\tt\fontsize{6.7pt}{8pt}\selectfont]{listings/01_add_reduction_v1_p02.txt}
\end{block}
On the \emph{left} the \alert{index} column contains the loop iteration counter and the index of the instruction in the block.\\
\medskip
On the \emph{right} the instructions are shown.\\
\medskip
In the \emph{middle} there is a \alert{timing graph} showing in which stage the instruction is in at each CPU cycle (from left to right). 
\end{frame}


\begin{frame}{The Timeline View}
\begin{block}{Timeline View}
\txtinput[\tt\fontsize{6.7pt}{8pt}\selectfont]{listings/01_add_reduction_v1_p02.txt}
\end{block}
\begin{description}
\item[\texttt{D}] The instruction is \alert{dispatched}\\(allocated to a reservation station)
\item[\texttt{e}] The instruction is being \alert{executed}
\item[\texttt{E}] The instruction \alert{finishes execution}, the result is ready
\item[\texttt{R}] The instruction is \alert{retired}\\(the reservation station is freed)
\end{description}
\end{frame}


\begin{frame}{The Instruction Info View}
\begin{block}{Instruction Info View}
\txtinput[\tt\fontsize{6pt}{7pt}\selectfont]{listings/01_add_reduction_v1_p03.txt}
\end{block}
\begin{itemize}
\item The dispatch width we saw earlier is \alert{6}
\item In the timeline view, at most \alert{3} instructions are dispatched together
\item In fact, can see in the \alert{instruction info view} that our instructions take 2 microinstructions to execute!
\end{itemize}
\end{frame}


\begin{frame}{The Instruction Info View}
\begin{block}{Instruction Info View}
\txtinput[\tt\fontsize{6pt}{7pt}\selectfont]{listings/01_add_reduction_v1_p03.txt}
\end{block}
Other interesting information shown here:
\begin{description}[\texttt{RThroughput}]
\item[\texttt{Latency}] Number of cycles required to execute the instruction
\item[\texttt{RThroughput}] Reciprocal of the number of such instructions that can execute simultaneously
\end{description}
\end{frame}


\begin{frame}{The Timeline and Microinstructions}
\begin{block}{Timeline View}
\txtinput[\tt\fontsize{6.7pt}{8pt}\selectfont]{listings/01_add_reduction_v1_p02.txt}
\end{block}
These two microinstructions (presumably!) perform, in order:
\begin{enumerate}
\item the \alert{load from memory} of the next number to accumulate
\item the \alert{sum} of the loaded number and the value in the register
\end{enumerate}
This is why the executions partially overlap --- only the \alert{sum} microinstruction is blocking.
\end{frame}


\begin{frame}{The Bottleneck Analysis}
\begin{overprint}
\onslide<1>
LLVM-MCA is able to automatically digest all this information for us, producing an analysis of the instruction sequence which has the biggest impact on the execution.
\onslide<2>
We can see right away that there is a data dependency on the \texttt{\%xmm0} register.
\end{overprint}
\begin{block}{Bottleneck View}
\txtinput[\tt\fontsize{5.7pt}{6pt}\selectfont]{listings/01_add_reduction_v1_p04.txt}
\end{block}
\end{frame}


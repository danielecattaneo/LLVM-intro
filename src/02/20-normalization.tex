% !TEX root = 02.tex

\section{Normalization Passes}


\begin{frame}{Canonicalize Pass Input}
We will see the following passes:

\begin{table}
\centering
\begin{tabular}{cc}
\toprule

\multicolumn{1}{c}{\textbf{Pass}}    &
\multicolumn{1}{c}{\textbf{Switch}} \\

\midrule

Variable promotion  &
\texttt{mem2reg}   \\

Loop simplify           &
\texttt{loop-simplify} \\

Loop-closed SSA  &
\texttt{lcssa}  \\

Induction variable simplification  &
\texttt{indvars}                  \\

\bottomrule
\end{tabular}
\end{table}

They are \alert{normalization} passes:

\begin{itemize}
\item they convert the code into a canonical form
\end{itemize}
\end{frame}


\begin{frame}{Variable Promotion}
\begin{center}
One of the most difficult things in compilers is\\\alert{handling memory accesses}.
\end{center}
\vfill
\begin{block}{Plain SAXPY (Scalar $ax + y$)}
\centering
\llvminput[\tt\footnotesize]{snippet/plain-saxpy.ll}
\end{block}
\end{frame}


\begin{frame}{Variable Promotion}{Simplifying Representation}
In the SAXPY kernel all the variables are \llvminline{alloca}ted on the stack!
\begin{itemize}
\item Function arguments included!
\end{itemize}

\vfill
They are allocated like that because the compiler follows a \alert{conservative} approach:
\begin{itemize}
\item an instruction could take the address of one of the variables...
\end{itemize}

\vfill
However, complex representations make optimizations more difficult:
\begin{itemize}
\item suppose you want to compute the \cinline{a * x + y} expression using only \alert{one}
      instruction (aka FMA4)
\item hard to detect due to \llvminline{load} and \llvminline{store}
\end{itemize}
\end{frame}


\begin{frame}{Variable Promotion}{Using Memory Only When Necessary}
To limit the number of instruction accessing memory we need to eliminate \llvminline{load} and \llvminline{store}
\begin{itemize}
\item achieved by \alert{promoting} variables from memory to registers
\end{itemize}

\vfill
Inside the LLVM-IR:
\begin{description}
\item[memory] Stack allocations \\
              \llvminline{\%1 = alloca float, align 4}
\item[register] SSA variables \\
							\llvminline{\%a}
\end{description}

\vfill
The \texttt{mem2reg} pass focus on:
\begin{itemize}
\item eliminating \llvminline{alloca} used only by \llvminline{load} and
      \llvminline{store} instructions
\end{itemize}

Also available as a utility function:
\begin{itemize}
\item \cppinline{llvm::PromoteMemToReg}\\
\begin{itemize}
\item see \texttt{llvm/Transforms/Utils/PromoteMemToReg.h}
\end{itemize}
\end{itemize}
\end{frame}


\begin{frame}{Variable Promotion}{Example on simplified code}
\begin{columns}[t]
\column{.45\textwidth}
\begin{block}{Starting Point}
\llvminput[\tt\footnotesize]{snippet/saxpy.ll}
\end{block}

(copy propagation is performed transparently by the compiler)

\column{.45\textwidth}
\begin{block}{Promoting \llvminline{alloca}}
\llvminput[\tt\footnotesize]{snippet/mem2reg-saxpy.ll}
\end{block}

\begin{block}{After Copy-propagation}
\llvminput[\tt\footnotesize]{snippet/mem2reg-copy-saxpy.ll}
\end{block}

\end{columns}
\end{frame}


\begin{frame}{Loops}
Different kind of loops:

\begin{columns}[t]
\column{.30\textwidth}
\begin{block}{\cinline{do}-\cinline{while} Loops}
\centering

% do-while-loop.tex: do-while loop shape.

\begin{tikzpicture}
[
  every node/.style={
    font=\tiny
  },
  every initial by arrow/.style={
    shorten >=.5mm
  },
  every accepting by arrow/.style={
    shorten <=.5mm
  },
  entry above/.style={
    initial above,
    initial text=
  },
  exit below/.style={
    accepting below
  },
  bb/.style={
    circle,
    draw
  },
  tip/.style={
    shorten <=.5mm,
    shorten >=.5mm,
    ->,
    draw
  }
]

\node (entry) [bb,entry above] {$1$};
\node (body)  [bb,exit below,below=4mm of entry] {$2$};

\path [tip, bend right] (entry) edge (body);
\path [tip, bend right] (body) edge (entry);
\end{tikzpicture}

\end{block}

\column{.30\textwidth}
\begin{block}{\cinline{while} Loops}
\centering

% while-loop.tex: while loop shape.

\begin{tikzpicture}
[
  every node/.style={
    font=\tiny
  },
  every initial by arrow/.style={
    shorten >=.5mm
  },
  every accepting by arrow/.style={
    shorten <=.5mm
  },
  entry above/.style={
    initial above,
    initial text=
  },
  exit below/.style={
    accepting below
  },
  bb/.style={
    circle,
    draw
  },
  tip/.style={
    shorten <=.5mm,
    shorten >=.5mm,
    ->,
    draw
  }
]

\node (entry) [bb,entry above,exit below] {$1$};
\node (body)  [bb,right=4mm of entry] {$2$};

\path [tip, bend right] (entry) edge (body);
\path [tip, bend right] (body) edge (entry);
\end{tikzpicture}

\end{block}

\column{.30\textwidth}
\begin{block}{Irreducible Loops}
\centering

% irreducible-loop.tex: irreducible loop shape.

\begin{tikzpicture}
[
  every node/.style={
    font=\tiny
  },
  every initial by arrow/.style={
    shorten >=.5mm
  },
  every accepting by arrow/.style={
    shorten <=.5mm
  },
  entry above/.style={
    initial above,
    initial text=
  },
  exit below/.style={
    accepting below
  },
  bb/.style={
    circle,
    draw
  },
  tip/.style={
    shorten <=.5mm,
    shorten >=.5mm,
    ->,
    draw
  }
]

\node (entry)  [bb,entry above] {$1$};
\node (body-1) [bb,exit below,below left=4mm of entry] {$2$};
\node (body-2) [bb,exit below,below right=4mm of entry] {$3$};

\path [tip,bend right] (entry) edge (body-1);
\path [tip,bend left]  (entry) edge (body-2);

\path [tip,bend right] (body-1) edge (body-2);
\path [tip,bend right] (body-2) edge (body-1);
\end{tikzpicture}

\end{block}
\end{columns}

\bigskip
In LLVM the focus is on one kind of loop:

\begin{itemize}
\item natural loops
\end{itemize}
\end{frame}

\begin{frame}{Natural Loops}
A natural loop:

\begin{itemize}
\item has only one entry node -- \emph{header}
\item there is a back edge that enter the loop header
\end{itemize}

\vfill
Under this definition:

\begin{itemize}
\item the irreducible loop is not a natural loop
\item since LLVM consider only natural loops, the irreducible loop \alert{is not
      recognized} as a loop
\end{itemize}
\end{frame}

\begin{frame}{Loop Terminology}
Loops defined starting from back-edges:

\vfill
\begin{description}
\item[back-edge] edge entering loop header: $(3,1)$
\end{description}

\begin{columns}[t]
\column{.59\textwidth}
\begin{description}
\item[header] loop entry node: $1$
\item[body] nodes that can reach back-edge source node ($3$) without passing
            from back-edge target node ($1$) plus back-edge target node:
            $\{1 ,2, 3\}$
\end{description}
\column{.32\textwidth}
\vspace*{-1em}
\begin{block}{\small A loop}
\vspace*{-1em}

% loop-nodes.tex: loop terminology by picture.

\begin{tikzpicture}
[
  every node/.style={
    font=\tiny
  },
  every initial by arrow/.style={
    shorten >=.5mm
  },
  every accepting by arrow/.style={
    shorten <=.5mm
  },
  entry above/.style={
    initial above,
    initial text=
  },
  entry left/.style={
    initial left,
    initial text=
  },
  exit below/.style={
    accepting below
  },
  bb/.style={
    circle,
    draw
  },
  tip/.style={
    shorten <=.5mm,
    shorten >=.5mm,
    ->,
    draw
  }
]

\node (entry)  [bb,entry above] {$1$};
\node (body-1) [bb,right=4mm of entry] {$2$};
\node (body-2) [bb,right=4mm of body-1] {$3$};

\node (exit-1) [bb,entry left,exit below,below=4mm of entry] {$4$};
\node (exit-2) [bb,exit below,below=4mm of body-2] {$5$};

\path [tip,bend left] (entry) edge (body-1);
\path [tip,bend left] (body-1) edge (body-2);

\path [tip,bend left]    (body-1) edge (entry);
\path [tip,bend left=60] (body-2) edge (entry);

\path [tip,bend right] (entry) edge (exit-1);
\path [tip,bend left]  (body-2) edge (exit-2);
\end{tikzpicture}

\end{block}


\end{columns}

\begin{description}
\item[exiting] nodes with a successor outside the loop: $\{1, 3\}$
\item[exit] nodes with a predecessor inside the loop: $\{4, 5\}$
\end{description}
\end{frame}

\begin{frame}{Loop Simplify}
Natural loops finding is the base pass \alert{identify} loops, but:

\begin{itemize}
\item some features are not analysis/optimization friendly
\end{itemize}

\vfill
The \texttt{loop-simplify} pass normalize natural loops:

\begin{columns}[t]
\column{.50\textwidth}
\begin{description}
\item[pre-header] the \alert{only predecessor} of \alert{header} node
\item[latch] the \alert{starting node} of the \alert{only back-edge}
\item[exit-block] ensures \alert{exits dominated} by loop \alert{header}
\end{description}

\column{.40\textwidth}
\begin{block}{Pre-header Insertion}
\centering

% pre-header.tex: adding pre-header.

\begin{tikzpicture}
[
  every node/.style={
    font=\tiny
  },
  every initial by arrow/.style={
    shorten >=.5mm
  },
  every accepting by arrow/.style={
    shorten <=.5mm
  },
  entry above/.style={
    initial above,
    initial text=
  },
  entry left/.style={
    initial left,
    initial text=
  },
  exit below/.style={
    accepting below
  },
  bb/.style={
    circle,
    draw
  },
  tip/.style={
    shorten <=.5mm,
    shorten >=.5mm,
    ->,
    draw
  }
]

\node (pre-header) [bb,entry above,fill=red!30] {$0$};

\node (entry)  [bb,below=4mm of pre-header] {$1$};
\node (body-1) [bb,right=4mm of entry] {$2$};
\node (body-2) [bb,right=4mm of body-1] {$3$};

\node (exit-1) [bb,entry left,exit below,below=4mm of entry] {$4$};
\node (exit-2) [bb,exit below,below=4mm of body-2] {$5$};

\path [tip,color=red] (pre-header) edge (entry);

\path [tip,bend left] (entry) edge (body-1);
\path [tip,bend left] (body-1) edge (body-2);

\path [tip,bend left]    (body-1) edge (entry);
\path [tip,bend left=60] (body-2) edge (entry);

\path [tip,bend right] (entry) edge (exit-1);
\path [tip,bend left]  (body-2) edge (exit-2);
\end{tikzpicture}

\end{block}
\end{columns}
\end{frame}

\begin{frame}{Loop Simplify}{Example}
\begin{columns}[t]
\column{.45\textwidth}
\begin{block}{Latch Insertion}
\centering

% latch.tex: ensure only one back-edge.

\begin{tikzpicture}
[
  every node/.style={
    font=\tiny
  },
  every initial by arrow/.style={
    shorten >=.5mm
  },
  every accepting by arrow/.style={
    shorten <=.5mm
  },
  entry above/.style={
    initial above,
    initial text=
  },
  entry left/.style={
    initial left,
    initial text=
  },
  exit below/.style={
    accepting below
  },
  bb/.style={
    circle,
    draw
  },
  tip/.style={
    shorten <=.5mm,
    shorten >=.5mm,
    ->,
    draw
  }
]

\node (pre-header) [bb,entry above,fill=red!30] {$0$};

\node (entry)  [bb,below=4mm of pre-header] {$1$};
\node (body-1) [bb,right=4mm of entry] {$2$};
\node (body-2) [bb,right=4mm of body-1] {$3$};
\node (latch)  [bb,above=4mm of body-1,fill=green!30] {$6$};

\node (exit-1) [bb,entry left,exit below,below=4mm of entry] {$4$};
\node (exit-2) [bb,exit below,below=4mm of body-2] {$5$};

\path [tip,color=red] (pre-header) edge (entry);

\path [tip,bend left] (entry) edge (body-1);
\path [tip,bend left] (body-1) edge (body-2);

\path [tip,color=green]            (body-1) edge (latch);
\path [tip,color=green,bend right] (body-2) edge (latch);
\path [tip,color=green,bend right] (latch) edge (entry);

\path [tip,bend right] (entry) edge (exit-1);
\path [tip,bend left]  (body-2) edge (exit-2);
\end{tikzpicture}

\end{block}

\column{.45\textwidth}
\begin{block}{Exit-block Insertion}
\centering

% exit-block.tex: ensure loop exit dominated by loop header.

\begin{tikzpicture}
[
  every node/.style={
    font=\tiny
  },
  every initial by arrow/.style={
    shorten >=.5mm
  },
  every accepting by arrow/.style={
    shorten <=.5mm
  },
  entry above/.style={
    initial above,
    initial text=
  },
  entry left/.style={
    initial left,
    initial text=
  },
  exit below/.style={
    accepting below
  },
  bb/.style={
    circle,
    draw
  },
  tip/.style={
    shorten <=.5mm,
    shorten >=.5mm,
    ->,
    draw
  }
]

\node (pre-header) [bb,entry above,fill=red!30] {$0$};

\node (entry)  [bb,below=4mm of pre-header] {$1$};
\node (body-1) [bb,right=4mm of entry] {$2$};
\node (body-2) [bb,right=4mm of body-1] {$3$};
\node (latch)  [bb,above=4mm of body-1,fill=green!30] {$6$};

\node (exit-0) [bb,below=4mm of body-1,fill=blue!30] {$7$};
\node (exit-1) [bb,entry left,exit below,below=4mm of entry] {$4$};
\node (exit-2) [bb,exit below,below=4mm of body-2] {$5$};

\path [tip,color=red] (pre-header) edge (entry);

\path [tip,bend left] (entry) edge (body-1);
\path [tip,bend left] (body-1) edge (body-2);

\path [tip,color=green]            (body-1) edge (latch);
\path [tip,color=green,bend right] (body-2) edge (latch);
\path [tip,color=green,bend right] (latch) edge (entry);

\path [tip,bend left,color=blue]  (entry) edge (exit-0);
\path [tip,bend right,color=blue] (exit-0) edge (exit-1);
\path [tip,bend left]             (body-2) edge (exit-2);
\end{tikzpicture}

\end{block}
\end{columns}

\begin{itemize}
\item pre-header always executed before entering the loop
\item latch always executed before starting a new iteration
\item exit-blocks always executed after exiting the loop
\end{itemize}
\end{frame}

\begin{frame}{Loop-closed SSA}
Loop representation can be further normalized:

\begin{itemize}
\item \texttt{loop-simplify} normalize the \alert{shape} of the loop
\item nothing is said about loop definitions
\end{itemize}

\vfill
Keeping SSA form is expensive with loops:

\begin{itemize}
\item \texttt{lcssa} insert \llvminline{phi} instruction at loop boundaries for
      variables \alert{defined inside} the loop body and \alert{used outside}
\item this guarantees isolation between optimization performed inside and outside
      the loop
\item faster keeping IR into SSA form -- propagation of code changes outside the
      loop blocked by \llvminline{phi} instructions
\end{itemize}
\end{frame}

\begin{frame}{Loop-closed SSA}{Example}
\begin{block}{Linear Search}
\centering
\cinput{snippet/lcssa.c}
\end{block}

\vfill
The example is trivial:

\begin{itemize}
\item think about having large loop bodies
\item transformation becomes useful
\end{itemize}
\end{frame}

\begin{frame}{Loop-closed SSA}{Example}
\begin{block}{Before LCSSA}
\centering
\llvminput{snippet/before-lcssa.ll}
\end{block}
\vspace{\baselineskip}
\vfill
\end{frame}

\begin{frame}{Loop-closed SSA}{Example}
\begin{block}{After LCSSA}
\centering
\llvminput{snippet/after-lcssa.ll}
\end{block}
\vfill
\end{frame}

\begin{frame}{Induction Variables}
Some loop variables are \emph{special}:

\begin{itemize}
\item e.g. counters
\end{itemize}

\vfill
Generalization lead to \alert{induction variables}:

\begin{itemize}
\item \cinline{foo} is a loop induction variable if its successive values form
      an arithmetic progression:

      \begin{center}
      \cinline{foo = bar * baz + biz}
      \end{center}

      where \cinline{bar, biz} are
      loop-invariant~\footnote{Constants inside the loop}, and \cinline{baz} is
      an induction variable
\item \cinline{foo} is a \alert{canonical} induction variable if it is always
      incremented by a constant amount:

      \begin{center}
      \cinline{foo = foo + biz}
      \end{center}

      where \cinline{biz} is loop-invariant
\end{itemize}
\end{frame}

\begin{frame}{Induction Variable Simplification}
Canonical induction variables are used to \alert{drive} loop execution:

\begin{itemize}
\item given a loop, the \texttt{indvars} pass tries to find its canonical
      induction variable
\end{itemize}

\vfill
With respect to theory, LLVM canonical induction variable is:

\begin{itemize}
\item initialized to \llvminline{0}
\item incremented by \llvminline{1} at each loop iteration
\end{itemize}
\end{frame}

\begin{frame}{Normalization}{Wrap-up}
Normalization passes running order:

\begin{enumerate}
\item \texttt{mem2reg}: limit use of memory, increasing the effectiveness of
       subsequent passes
\item \texttt{loop-simplify}: canonicalize loop shape, lower burden of writing
      passes
\item \texttt{lcssa}: keep effects of subsequent loop optimizations local,
      limiting overhead of maintaining SSA form
\item \texttt{indvars}: normalize induction variables, highlighting the
      canonical induction variable
\end{enumerate}

\vfill
Other normalization passes available:

\begin{itemize}
\item try running \texttt{opt -help}
\end{itemize}
\end{frame}


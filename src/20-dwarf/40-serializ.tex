% !TEX root = main.tex

\section{Data Representation}


\begin{frame}{TODO}
\begin{itemize}
\item Sections used by DWARF
\item Representing DIEs (info and abbrev tbl)
\item Representing line number tbl
\end{itemize}
\end{frame}


\begin{frame}{The physical reality (?) of DWARF}
DWARF debug info is scattered in different sections, whose names all start with \texttt{.debug}:\\
\bigskip
\begin{tabular}{ l l }
\texttt{.debug\_info}, \texttt{.debug\_abbrev} & DIE tree \\
\texttt{.debug\_line} & Line number table\\
\texttt{.debug\_frame} & Call Frame Information \\
\texttt{.debug\_macro} & Macro information \\
\texttt{.debug\_pubnames},\texttt{.debug\_pubtypes}  & Index \\
\end{tabular}\\
\end{frame}


\begin{frame}{The physical reality (?) of DWARF}
The information for some structures is \alert{scattered further} (for a mix of backwards
compatibility and data compactness reasons) into \alert{more sections}:\\
\bigskip
\begin{tabular}{ l l }
\texttt{.debug\_addr} & Constant pool for DWARF expressions\\
\texttt{.debug\_aranges} & Address ranges for compilation units\\
\texttt{.debug\_line\_str} & String pool for \texttt{.debug\_line}\\
\texttt{.debug\_loc} & Location lists addressed by \texttt{.debug\_info}\\
\texttt{.debug\_ranges} & Range list addressed by \texttt{.debug\_loc}\\
\texttt{.debug\_str} & String pool for \texttt{.debug\_info}\\
\ldots & \ldots
\end{tabular}\\
\end{frame}


\begin{frame}{A simple and logical organization}
\centering
\includegraphics[height=0.75\textheight]{img/sec_relations.pdf}
\end{frame}


\begin{frame}{A simple and logical organization}
We will focus on:
\begin{itemize}
\item Representation of the main DIE structure
\item The line number table
\end{itemize}
\bigskip
These two structures are the most important anyway.
\end{frame}


\begin{frame}{The \texttt{.debug\_info} section}
This section contains the main structure of DWARF: the tree of Debug
Information Entries.\\
\medskip
It contains:
\begin{enumerate}
\item An header
\item The serialized DIEs, enumerated in depth-first order
	\begin{itemize}
	\normalsize
	\item Each DIE has a flag specifying if it has children
	\item A special flag marks the end of a chain of children
	\end{itemize}
\end{enumerate}
\bigskip
This structure is repeated for each compilation unit in the file
(in the case of a non-linked object file there 
is always just one compilation unit)
\end{frame}


\begin{frame}{The \texttt{.debug\_info} section}{The Header}
The header of a \texttt{.debug\_info} is simple:\\
\medskip
\begin{tabular}{ l l }
\texttt{uint32\_t length;} & Size of the info for this unit \\
\texttt{uint16\_t version;} & DWARF version\\
\texttt{uint32\_t abbrev\_offset;} & Offset in the \alert{abbreviation table} \\
\texttt{uint8\_t address\_size;} & Size of a pointer (in bytes) \\
\end{tabular}\\
\bigskip
Wait... what's an \alert{abbreviation table}?
\end{frame}


\begin{frame}{The \texttt{.debug\_abbrev} section}
The \alert{abbreviation table} is the main element of the \alert{compression scheme}
used by DWARF to represent DIEs.\\
\medskip
\begin{itemize}
\item The typical DIEs produced by a compiler have similarities
	\begin{itemize}
	\item The same type of DIE often has the same type of attributes
	\item Similarly structured DIEs will have different attributes values
	\end{itemize}
\item The format of a DIE may change in the future
	\begin{itemize}
	\item We need to allow old implementations to read new DWARF records without
		getting confused
	\end{itemize}
\end{itemize}
\medskip
The abbreviation table \alert{decouples the tag, the attributes, the order of
appearance of the attributes, and the type of each attribute in the DIE
from the values of the attributes}
\end{frame}


\begin{frame}{The \texttt{.debug\_abbrev} section}
The abbreviation table contains a flat list of DIE patterns, in the following format:

\begin{columns}[onlytextwidth]

\column{0.55\textwidth}
\begin{block}{Abbreviation table in pseudo-C}
\cinput{listings/abbrev-tbl.c}
\end{block}

\column{0.4\textwidth}
\begin{itemize}
\item M is the number of attribute-value pairs in the abbreviated DIE,
	it changes for each abbreviation
\item N is the total number of abbreviations
\end{itemize}

\end{columns}

\end{frame}


\begin{frame}{The \texttt{uleb128} format}
\alert{uleb128} is a \alert{variable length numeric representation} used in DWARF
for compactness.\\
\medskip
It reduces the number of leading zero bytes in the representation by using bit 7
as a stop marker. Maximum length of 128 bits gives it its name.\\
\medskip
\begin{itemize}
\item Bit 7 = 0 $\Rightarrow$ This is the last and most significant byte
\item Bit 7 = 1 $\Rightarrow$ This is not the last byte, read another one
\end{itemize}
\medskip
\begin{block}{uleb128 example}
\tt
0x0001 = 0b0000000\_0000000\_0000001 = 0x01\\
0x007F = 0b0000000\_0000000\_1111111 = 0x7F\\
0x0080 = 0b0000000\_0000001\_0000000 = 0x80 0x01\\
0xAA55 = 0b0000010\_1010100\_1010101 = 0xD5 0xD4 0x02 \\
\end{block}
\end{frame}


\begin{frame}{The \texttt{.debug\_info} section}{The Body}
Back to the \texttt{.debug\_info} section...\\
\medskip
After the header, for each DIE the section contains:
\begin{itemize}
\item The identifier of the matching abbreviation, encoded as \texttt{uleb128}
\item The values of the attributes, in the same order as in the abbreviation
	\begin{itemize}
	\item Not all attribute types use the same number of bytes! There is no alignment,
	  and some types use \alert{zero bytes}.
	\end{itemize}
\end{itemize}
\medskip
Abbreviation number zero does not exist; zero is used as the \alert{end-of-children
mark}.\\
\smallskip
Whether a DIE has children or not, it's specified in the abbreviation!
\end{frame}


\begin{frame}{The \texttt{.debug\_line} section}
In constrast with the \texttt{.debug\_info} section, the \texttt{.debug\_line} section
is much simpler. It encodes the \alert{line number table}.\\
\medskip
It begins with a header specifying:
\begin{itemize}
\item Length, version (like \texttt{.debug\_info})
\item Whether by default each line contains a statement (usually true)
\item The list of include directories used by the compiler
\item The list of files that are referenced by the line number table
\end{itemize}
(not an exhaustive list)\\
\medskip
The list of files and include directories are specified as offsets of strings in
the \texttt{.debug\_line\_str} section.
\end{frame}


\begin{frame}{The \texttt{.debug\_line} section}
After the header, the \alert{line number program} begins.\\
\medskip
Every instruction can either:
\begin{itemize}
\item Modify a cell in the current row of the table
	\begin{itemize}
	\item Each opcode corresponds to a different column
	\item \texttt{DW\_LNS\_advance\_pc}, \texttt{DW\_LNS\_advance\_line},
		\texttt{DW\_LNS\_set\_file}, ... 
	\end{itemize}
\item Create the next row by (mostly) copying the next row
	\begin{itemize}
	\item \texttt{DW\_LNS\_copy}
	\item The \textbf{Basic Block}, \textbf{Prologue End}, \textbf{Epilogue Begin} and \textbf{Discriminator} rows are set to zero regardless
	\end{itemize}
\medskip
\item Instructions either take no operands, or a single \texttt{uleb128}.
\end{itemize}
\end{frame}



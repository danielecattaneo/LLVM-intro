% !TEX root = main.tex

\section{Describing Code}


\begin{frame}{TODO}
\begin{itemize}
\item Overview of program structs (module = compilation unit, functions/subprograms, line number tbl)
\item DIEs for module and funcs
\item the line number tbl 
\end{itemize}
\end{frame}


\begin{frame}{Describing Code}
We have seen how DWARF describes \alert{data}.
Of equal importance in a debugger is analyzing the \alert{code being executed}.\\
\medskip
Obvious requirements:\\
\begin{itemize}
\item Map to instruction to line of code in the source code
	\begin{itemize}
	\item Essential for breakpoints, single stepping
	\end{itemize}
\end{itemize}
\medskip
Less obvious requirements:
\begin{itemize}
\item Map from code spans to compilation units
\item Map from code spans to functions
\item Identification of every function call in a function
	\begin{itemize}
	\item Used for breakpoints on functions, to parse backtraces, to resolve imported functions
	\end{itemize}
\item Grouping of lexical blocks in a function
	\begin{itemize}
	\item Used to determine the scope of variables
	\end{itemize}
\end{itemize}
\end{frame}


\begin{frame}{Describing Code}
\begin{itemize}
\item Compilation units, functions and function calls are part of the DIE tree
	\begin{itemize}
	\item Just like a compiler AST!
	\end{itemize}
\medskip
\item The line number table is its own separate structure
	\begin{itemize}
	\item This helps saving space inside the executable, as DIEs are not as space-efficient
	\end{itemize}
\end{itemize}
\end{frame}


\begin{frame}{Describing Code: Compilation Units}
A compilation unit (\texttt{DW\_TAG\_compile\_unit}) represents the result of a single compilation process (duh).\\
\medskip
Common attributes:\\
\medskip
\begin{tabular}{ l l }
\texttt{DW\_AT\_name} & File name that was compiled \\
\texttt{DW\_AT\_comp\_dir} & Working directory where the compiler\\
& was invoked \\
\texttt{DW\_AT\_language} & Language of the source code file \\
\texttt{DW\_AT\_stmt\_list} & Pointer to the line number table for this file \medskip \\

\texttt{DW\_AT\_low\_pc} & Range of addresses of the \\
\texttt{DW\_AT\_high\_pc} & compilation unit's code \\
\end{tabular}\\
\end{frame}


\begin{frame}{Describing Code: Functions}
A subprogram (\texttt{DW\_TAG\_subprogram}) represents a function. Function arguments are
described by children \texttt{DW\_TAG\_formal\_parameter} DIEs.\\
\medskip
Common attributes:\\
\medskip
\begin{tabular}{ l l }
\texttt{DW\_AT\_name} & Function name (as it appears in the source) \\
\texttt{DW\_AT\_type} & Type of the return value \\
\texttt{DW\_AT\_frame\_base} & Expression establishing the address of \\
& the frame pointer (\texttt{DW\_OP\_fbreg}) \\
\texttt{DW\_AT\_low\_pc} & \multirow{2}{*}{Range of addresses of the function's code} \\
\texttt{DW\_AT\_high\_pc} &  \\
\end{tabular}\\
\end{frame}


\begin{frame}{Describing Code: Blocks}
\texttt{DW\_TAG\_lexical\_block}s DIEs represent a block, in languages that support
this concept. Not strictly required, but helps the debugger in handling local variables
with a restricted scope.\\
\medskip
Common attributes:\\
\medskip
\begin{tabular}{ l l }
\texttt{DW\_AT\_low\_pc} & \multirow{2}{*}{Range of addresses of the function's code} \\
\texttt{DW\_AT\_high\_pc} &  \\
\end{tabular}\\
\smallskip
And that's it!\\
\bigskip
Note that every time the \texttt{DW\_AT\_low\_pc} and \texttt{DW\_AT\_high\_pc} attributes
are used together, they can be replaced by a single \texttt{DW\_AT\_ranges} attribute.\\
\smallskip
\texttt{DW\_AT\_ranges} supports describing multiple discontiguous ranges.
\end{frame}


\begin{frame}{Describing Code: Line Numbers}
The \alert{line number table} is its own data structure.\\
\begin{itemize}
\item Maps each instruction to a line number and more
\item Consecutive instructions mapping to the same line number are
	represented by just the first instruction
\item Internal representation consists of a \alert{line number program} 
	that builds the table (and allows for compact representation of
	similar rows)
	\begin{itemize}
	\item We will enter into details about this representation later
	\end{itemize}
\end{itemize}
\medskip
The GNU assembler supports building this data automatically using specific 
\texttt{.loc} directives.
\end{frame}


\begin{frame}{Describing Code: Line Numbers}
%The columns in the table are:
\begin{description}[Epilogue Begin]
\item[Address] Address of the instruction
\item[Op. Index] 0 for non-VLIW architectures
\item[File] File where the line is found (index in a list of files)
\item[Line] Line in the file (1-based)
\item[Column] Column in the file (1-based)
\item[Is Statement] True if position is at the beginning of a statement
\item[Basic Block] True if at the beginning of a basic block
\item[End Sequence] True in the last row of the table (ending marker)
\item[Prologue End] True in the last instruction of a function prologue
\item[Epilogue Begin] True in the first instruction of a function epilogue
\item[ISA] The instruction set of the code
\item[Discriminator] Disambiguates multiple blocks on the same line
\end{description}
\end{frame}


% !TEX root = main.tex

\section{Other information in DWARF}


\begin{frame}{TODO}
\begin{itemize}
\item Call Frame Info (and its applic in exc handling impl)
\item Macro Info
\item Accelerated Access
\end{itemize}
\end{frame}


\begin{frame}{Other information in DWARF}
DWARF also provides some extra useful but not essential information:
\begin{itemize}
\item Call Frame Information
\item Macro information (!)
\item Identifier index for faster lookups
\end{itemize}
\end{frame}


\begin{frame}{Call Frame Information}
DWARF Call Frame Information (aka CFI) allows the debugger to reliably perform
backtraces.\\
\medskip
Why we need it?\\
\medskip
\begin{enumerate}
\item The ABI already provides a way to perform backtraces without it, but it
	prevents optimizations like tail calls or (in the x86 architecture) using the
	EBP register for other purposes
\item DIEs by themselves do not contain any information about the size
	of the activation record of each function, just how to access variable values
\end{enumerate}
\end{frame}


\begin{frame}{Call Frame Information}
DWARF defines an activation record as the following set of values:\\
\medskip
\begin{itemize}
\item The return address (for past frames) or the current PC (for the current frame)
\item The area of memory in the stack corresponding to the call frame of the function
	(described as a single base pointer: the Canonical Frame Address)
\item The set of registers that are preserved across the call correponding to the
	activation record
\end{itemize}
\end{frame}


\begin{frame}{Call Frame Information}
The CFI maps each instruction in the program to a table
that specifies:\\
\medskip
\begin{itemize}
\item how to reconstruct each register value from the activation record
	alone, as it appears in that point of the subroutine
\item where the Canonical Frame Address pointer is
\end{itemize}
\bigskip
This information is compressed using a bytecode system similar to DWARF
expressions and the line number table.\\
\medskip
The GNU assembler supports building this data inline (like the line number table)
using specific \texttt{.cfi\_XXX} directives.\\
\bigskip
\textbf{Fun fact:} the same tables used by DWARF are used as a method for implementing
stack unwinding for C++ exceptions (hence why they are so slow!)
\end{frame}


\begin{frame}{Macro Information}
You wish you could \alert{debug your C macros}?\\
\medskip
DWARF includes a \alert{Macro Information} section which specifies to the debugger
where a macro declaration can be found in the source.\\
\medskip
There is nothing else in there... which limits its usefulness. It's also
100\% optional.\\
\medskip
The format of macro information has changed in DWARF 5.
\end{frame}


\begin{frame}{Accelerated Access}
DWARF has a facility for building an index of symbols to make lookup of
large DIE trees easier. It's implemented as a hashtable.\\
\medskip
The index can contain references to:
\begin{itemize}
\item Subprograms
\item Labels
\item Variables
\item Types
\item Namespaces
\end{itemize}
(we didn't see some of these DIEs earlier, but they exist)\\
\medskip
The index must be complete, this simplifies the lookup logic in case it exists
(it's optional).\\
\medskip
Just like with macro information, the format of the index has changed in DWARF 5.
\end{frame}


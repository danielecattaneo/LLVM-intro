% !TEX root = main.tex

\section{Describing Data}


\begin{frame}{Describing Data}
\begin{center}
A debugger like \texttt{gdb} is capable of \alert{evaluating expressions}.\\
\medskip
To do that, the debugger must be able to resolve each symbol to its current memory location,
in order to read its value.\\
\bigskip
\large
This is not as simple as it sounds!
\end{center}
\end{frame}


\begin{frame}{Where is my data?}
\begin{itemize}
\item Programming language symbols $\ne$ symbols in executable!
\medskip
	\begin{itemize}
	\normalsize
	\item Even though ELF already includes symbol names for several objects, they are meant for the \alert{linker}, and are not useful to the debugger!
	\item Names might have changed (some platform prepend an underscore to all symbol names)
	\item \texttt{static} globals do not have a symbol name
	\item Stack allocated variables are not symbolicated at all (the linker does not care about them)
	\medskip
	\item[$\Rightarrow$] DWARF must contain additional symbolication info!
	\end{itemize}
\end{itemize}
\end{frame}


\begin{frame}{Where is my data?}
\begin{itemize}
\item The debugger does not have any intrinsic knowledge of the data types being used
\medskip
	\begin{itemize}
	\normalsize
	\item Obvious example: custom \texttt{struct}s
	\item Non-obvious examples: floating point formats, big-endian vs. little-endian ints
	\medskip
	\item[$\Rightarrow$] DWARF must describe in detail all data types!
	\end{itemize}
\medskip
\item A variable associated to a symbol may be moved to \alert{different locations} during the execution of the program!
\medskip
	\begin{itemize}
	\normalsize
	\item A frequently-used variable may be moved from the stack to a register, and then back to the stack
	\medskip
	\item[$\Rightarrow$] Each variable location must be relative to a certain region of code!
	\end{itemize}
\end{itemize}
\end{frame}


\begin{frame}{Data Types in DWARF}
To support all this, DWARF uses specific kinds of DIE:\\
\medskip
\begin{tabular}{ll}
\texttt{DW\_TAG\_base\_type} & A base type (int, char, float...) \\
\texttt{DW\_TAG\_unspecified\_type} & An unknown type (void) \\
\texttt{DW\_TAG\_typedef} & A renamed type \\
\texttt{DW\_TAG\_pointer\_type} & A pointer to another type \\
\texttt{DW\_TAG\_array\_type} & An array \\
\texttt{DW\_TAG\_structure} & A structure \\
\texttt{DW\_TAG\_union} & An union \\
... & ...\\
\end{tabular}\\
\bigskip
Let's look at some examples...\\
\end{frame}


\begin{frame}{Data Types in DWARF}
\begin{columns}

\column{0.5\textwidth}
\begin{block}{Integer scalar}
\txtinput[\tt\scriptsize]{listings/type-int.txt}
\end{block}
\begin{block}{Integer pointer}
\txtinput[\tt\scriptsize]{listings/type-ptr.txt}
\end{block}

\column{0.5\textwidth}
\begin{block}{Integer array}
\txtinput[\tt\scriptsize]{listings/type-array.txt}
\end{block}

\end{columns}
\end{frame}


\begin{frame}{Data Types in DWARF}
\begin{centering}
Additional attributes (\texttt{volatile}, \texttt{const}...)\\
are represented by separate types\\
\medskip
These separate types always point to the base type\\
(no parent-child relationships)\\
\end{centering}
\medskip
\begin{block}{Constant integers}
\txtinput[\tt\small]{listings/type-const-int.txt}
\end{block}
\end{frame}


\begin{frame}{Variables in DWARF}
As opposed to types, \alert{variables} \& co. are represented by just 3 DIE classes:\\
\medskip
\begin{tabular}{ l l }
\texttt{DW\_TAG\_variable} & A variable \\
\texttt{DW\_TAG\_formal\_parameter} & A function parameter \\
	& (i.e. inside the function) \\
\texttt{DW\_TAG\_constant} & A constant \\
\end{tabular}\\
\bigskip
Conceptually, there's nothing special here as well...
\end{frame}


\begin{frame}{Variables in DWARF}
\begin{columns}

\column{0.5\textwidth}
\begin{block}{Global integer scalar}
\txtinput[\tt\scriptsize]{listings/var-g-int.txt}
\end{block}
\begin{block}{Local integer scalar}
\txtinput[\tt\scriptsize]{listings/var-l-int.txt}
\end{block}

\column{0.5\textwidth}
\begin{itemize}
\item Apart from the obvious \texttt{name} and \texttt{type} attributes,
the \texttt{external} and \texttt{location} attributes appeared
\item \texttt{external} specifies if the symbol is visible from outside the compilation unit
\item \texttt{\textbf{location}} specifies where the data is located using a \alert{DWARF expression}
\end{itemize}

\end{columns}
\end{frame}


\begin{frame}{DWARF expressions}
\begin{itemize}
\item DWARF expressions are composed of \alert{sequences of commands}
\item They are interpreted by a simple \alert{stack-based evaluator}
\smallskip
\item Commands can optionally take parameters, use \alert{prefix notation}
\item There are commands for branches, function calls...
\item The values in the evaluator are either \alert{void pointers} or \alert{base types} supported by the target machine
\item All results of commands are pushed on top of the stack
\smallskip
\item Commands are evaluated in the order they appear (bar jumps / calls)
\item It's a mini assembly language!
\end{itemize}
\end{frame}


\begin{frame}{DWARF expressions}
DWARF expressions are powerful, but this power is seldom used or needed.\\
\smallskip
Most used commands:\\
\medskip
\begin{tabular}{ l l }
\texttt{DW\_OP\_fbreg \textit{x}} & The value of the current frame pointer plus \textit{x} \\
\texttt{DW\_OP\_reg\textit{n}} & The location of register \textit{n} \\
& (0 to 31, represented in an opaque encoding) \\
\texttt{DW\_OP\_addr \textit{x}} & The absolute address \textit{x} \\
\end{tabular}\\
\bigskip
\begin{center}
Remember: the DWARF data sections are\\\emph{linked together with the rest of the program}\\
$\Downarrow$\\
Any address that appears in them can be fixed up\\by a static or a dynamic linker!
\end{center}
\end{frame}


\begin{frame}{DWARF Location Descriptors}
Actually, \texttt{DW\_AT\_location} attributes can be associated to two kinds of values:\\
\medskip
\begin{enumerate}
\item One DWARF expression describing where the data is allocated\\
	(\emph{Single location descriptions})\\ \smallskip
	Used when the data is always allocated in the same place
\medskip
\item A map between code ranges and DWARF expressions\\
	(\emph{Location lists})\\ \smallskip
	Used when the data changes its location during its lifetime
\end{enumerate}
\end{frame}


\begin{frame}{DWARF Location Descriptors}
\begin{columns}

\column{0.5\textwidth}
\begin{block}{C code}
\cinput[\tt\scriptsize]{listings/var-l-move.c}
\end{block}
\begin{block}{Debug Info}
\txtinput[\tt\scriptsize]{listings/var-l-move.txt}
\end{block}

\column{0.5\textwidth}
\begin{block}{Assembly Code (-g -O2)}
\asminput[\tt\scriptsize]{listings/var-l-move.s}
\end{block}

GCC optimized out the \texttt{i} variable and moved \texttt{p}
to two different registers.

\end{columns}
\end{frame}


\begin{frame}{DWARF Declaration Coordinates}
All of these DIEs (along with many others) support specifying where each type is declared
using specific attributes:\\
\medskip
\begin{tabular}{ l l }
\texttt{DW\_AT\_decl\_file} & File where the declaration was found \\
\texttt{DW\_AT\_decl\_line} & Line number (1-based) of the declaration \\
\texttt{DW\_AT\_decl\_column} & Column number (1-based) \\
\end{tabular}\\
\medskip
\begin{block}{Declaration Coordinates}
\txtinput[\tt\small]{listings/var-l-int-declcoord.txt}
\end{block}
\end{frame}




% !TEX root = main.tex

\section{Conclusion}


\begin{frame}{Conclusion}
We have seen:
\begin{itemize}
\item A brief history of debugging formats, why they are needed, what DWARF is
\item What is the big idea behind DWARF
\item How DWARF is actually represented inside an object file
\end{itemize}
\end{frame}


\begin{frame}{Conclusion}
With this knowledge you could:

\begin{itemize}
\item Build your own compiler from scratch with debugging info support
\item Build your own source-level debugger
\item Improve debugging support in other compilers
\end{itemize}
\end{frame}


\begin{frame}{One last note!}
If you are a LLVM user... the implementation details are \alert{abstracted} from you!\\
\medskip
LLVM represents debug information using a more generic data structure known
as \alert{metadata}.\\
\medskip
Debug information is important, and it is \alert{up to the passes} to mantain its
correctness, so keep this in mind when you work with LLVM.\\
\bigskip
LLVM documentation about debugging information: \url{https://llvm.org/docs/SourceLevelDebugging.html}
\end{frame}


\begin{frame}{I want to know more about dwarves!}
\Large
\centering
\vfill
\url{http://dwarfstd.org}\\
\vfill
\scriptsize
Or get in contact with your local DM \texttt{;)}
\end{frame}


\begin{frame}[plain]{}
\Huge\centering
Thank You!\\
\bigskip
\normalsize
Questions?
\end{frame}



% !TEX root = main.tex

\section{Inside LLVM}


\begin{frame}{Using LLVM}
\begin{center}
LLVM is a \alert{compiler construction framework}\\
It operates on the \alert{LLVM-IR} language.\\
\bigskip
$\Downarrow$\\
\bigskip
Using LLVM \emph{by itself} does not make much sense!\\
Writing LLVM-IR by hand is unfeasible.
\end{center}
\end{frame}


\begin{frame}{Terminology}{Speaking About LLVM IR}
\begin{center}
LLVM-IR comes in 3 different flavours:
\end{center}
\bigskip
\begin{description}
\item[assembly] on-disk human-readable format\\(file extension: \texttt{.ll})
\item[bitcode] on-disk machine-oriented binary format\\(file extension: \texttt{.bc})
\item[in-memory] in-memory binary format\\(used during compilation process)
\end{description}
\bigskip
\begin{center}
All formats have the same expressiveness!
\end{center}
\vfill
\end{frame}


\begin{frame}{Frontends and Drivers}{C Language Family Front-end}
The LLVM-IR input to LLVM is provided by \alert{frontends}.
\begin{example}
\alert{Clang}\cite{LOCAL:www/clang} is the frontend for the C language family
\smallskip % why does \cite screw up the f*ing height of my block!?
\end{example}

The \alert{compiler driver} is the program that:
\begin{itemize}
\item Provides the interface to the user
\item Performs setup of the front end and LLVM itself. 
\end{itemize}

\begin{example}
The driver of \emph{Clang} is the \texttt{clang} executable (compatible with GCC)
\end{example}
\end{frame}


\begin{frame}{Frontends and Drivers}{Using the driver to produce LLVM-IR}
\begin{center}
\vfill
We can generate LLVM-IR assembly using the \texttt{clang} driver:\\
\bigskip
\texttt{clang -emit-llvm -S -o out.ll in.c}\\
\medskip
{\footnotesize If you want to generate bitcode instead:\\
\texttt{clang -emit-llvm -o out.bc in.c}\\
}
\bigskip
The compiler driver can also generate native code starting from 
LLVM-IR assembly\\
\smallskip
{\small(Like compiling an assembly file with GCC)}
\vfill
\end{center}
\end{frame}


\begin{frame}{Tools}{Playing with LLVM Passes}
Run one or more passes on the LLVM-IR on-demand by using \texttt{opt}:

\begin{itemize}
\item Syntax is like \texttt{clang} (supports even \texttt{-O1}, \texttt{-O2}...)
\item One command line argument per pass to run
\item Order of execution is the same as the argument order
\begin{itemize}
\item Different order, different results! (\alert{phase/stage ordering})
\end{itemize}
\end{itemize}

\vfill
Some useful passes for debugging (they do not transform anything):

{\small
\begin{description}[print dominator tree]
\item[print CFG] \texttt{opt -view-cfg input.ll}
\item[print dominator tree] \texttt{opt -view-dom input.ll}
\item[print current IR] \texttt{opt -print-module input.ll}
\end{description}
}

\vfill
\begin{example}
\begin{itemize}
\item Run \emph{mem2reg}, then view the CFG:
\begin{itemize}
\item \texttt{opt -mem2reg -view-cfg input.ll}
\end{itemize}
\end{itemize}
\end{example}
\end{frame}


\begin{frame}{Pass Hierarchy}
LLVM provides a lot of passes...

\begin{itemize}
\item Try \texttt{opt -help}!
\end{itemize}

\vfill
For performance reasons there are different kind of passes:

\begin{block}{LLVM Passes}

% llvm-passes: hierarchy of LLVM passes.

\begin{tikzpicture}
[
  every node/.style={
    font=\scriptsize
  },
  extends/.style={
    draw,
    open triangle 90-
  },
  level distance=10mm,
  level 1/.style={
    sibling distance=20mm
  },
  level 3/.style={
    sibling distance=20mm
  }
]

\node {Pass}
  [edge from parent fork down,
   edge from parent/.style={extends}]

  child { node {CallGraphSCCPass}}
  child { node {ModulePass}
    child { node {ImmutablePass}}
  }
  child { node {FunctionPass}}
  child {node {LoopPass}}
  child {node {BasicBlockPass}};
\end{tikzpicture}

\centering
\end{block}
\end{frame}


\begin{frame}{LLVM Passes}
Each kind of pass visits particular elements of a module:

\begin{description}[align=left, labelwidth=1cm]
\item[ImmutablePass] compiler configuration -- never run
\item[CallGraphSCCPass] post-order visit of CallGraph SCCs
\item[ModulePass] visit the whole module
\item[FunctionPass] visit functions
\item[LoopPass] post-order visit of loop nests
%\item[BasicBlockPass] visit basic blocks % DEPRECATED AND REMOVED
\item[RegionPass] visit a custom-defined region of code
\end{description}

\vfill
Specializations come with restrictions:

\begin{itemize}
\item e.g. a \alert{FunctionPass} cannot add or delete functions
\item refer to ``Writing a LLVM Pass''~\cite{LOCAL:www/llvmWritingAPass}
      for documentation on features and limitations of each kind of pass
\end{itemize}
\end{frame}


% \begin{frame}{Examples}
% Now we will see very simple passes:
%
% \begin{itemize}
% \item some of them are meaningless
% \item goal is to show you the LLVM API
% \end{itemize}
%
% \vfill
%
% The passes are:
% \begin{description}
% \item[instruction-count] simple instruction counting analysis
% \item[hello-llvm] optimization pass building an hello-world program
% \item[function-eraser] optimization pass removing ``small'' functions
% \end{description}
%
% \vfill
% Hint: take the LLVM pass writing tutorial~\cite{LOCAL:www/llvmWritingAPass}
% \end{frame}


\begin{frame}{Recap}
\begin{itemize}
\item The \alert{user} invokes the \alert{compiler}
\item The \alert{compiler} is made of three \alert{stages}
\item Each \alert{stage} is made of \alert{passes}
\end{itemize}
\medskip
If you want things done, you want to work on a \alert{pass}.\\
\medskip
\begin{itemize}
\item Edit an existing pass
\item Create a new pass
\end{itemize}
\medskip
But the question now becomes: how do passes work?
\end{frame}


% !TEX root = main.tex

\subsection{The LLVM-IR language}


\begin{frame}{How passes work}
\centering
A \alert{pass} is a \alert{subroutine} that \\programmatically
transforms a piece of code.\\
\bigskip
The code of the pass operates on the LLVM-IR \\using a set of
\alert{object-oriented APIs}.
\vfill
Let's examine the LLVM-IR~\cite{LOCAL:www/llvmLanguageRef} more closely,\\
first by looking at its \alert{human-readable} form.
\end{frame}


\begin{frame}{LLVM-IR}{Example: factorial}
\begin{center}
\begin{varwidth}{20cm}
\llvminput[\ttfamily\fontsize{7pt}{5pt}\selectfont]{snippet/fact.ll}
\end{varwidth}
\end{center}
\end{frame}


\begin{frame}{LLVM-IR}{What it looks like}
LLVM-IR looks a lot like a RISC assembly language:\\

\begin{itemize}
\item Few instructions, all perfectly orthogonal
	\begin{itemize}
	\item There are infinite registers
	\item There are no special-purpose registers
	\item No implicit flags register
	\end{itemize}
\item Basic block boundaries are denoted by \alert{labels}
\item Only \llvminline{load} and \llvminline{store} access memory
\end{itemize}

\vfill
There are also a few CISC-like \alert{high level instructions}:

\begin{itemize}
\item Reserve memory on the stack -- \llvminline{alloca}
\item Function call -- \llvminline{call}
	\begin{itemize}
	\item The calling convention is abstracted away
	\item There is an implicit call stack
	\end{itemize}
\item Pointer arithmetics -- \llvminline{getelementptr}
\item \ldots
\end{itemize}
\end{frame}


\begin{frame}{LLVM-IR}{How it is actually structured}
In reality LLVM-IR is much more high-level than assembly.
\begin{itemize}
\item The topmost object of a LLVM-IR program is the \alert{Module}.
\item \alert{Modules} contain a list of \alert{Globals}.
	\begin{itemize}
	\item {Globals} can be either \alert{Functions} or \alert{Global Variables}.
	\item A global can be a \alert{Forward declaration}.
	\end{itemize}
\item \alert{Functions} contain a list of \alert{Basic Blocks} + \alert{Arguments}.
\item \alert{Basic Blocks} are a list of \alert{Instructions}.
\end{itemize}
\begin{block}{The in-memory representation}
All these parts will correspond directly to \alert{C++ objects}.\\
The abundance of lists guarantees low overhead and scalability to very large programs.
\end{block}
\end{frame}


\begin{frame}{LLVM-IR}{Values \& Types}
LLVM-IR is \alert{strongly typed}:

\begin{itemize}
\item e.g. you cannot assign a floating point value to an integer register
without an explicit cast
\end{itemize}
\medskip
\alert{Almost everything} is \alert{typed}:\\
\medskip
\begin{tabular}{>{\RaggedLeft\arraybackslash}p{5.55em}lcl}
\textbf{functions} & \llvminline{@fact} & $\rightarrow$ & \llvminline{i32 (i32)} \\
\textbf{registers} & \llvminline{\%3 = icmp eq i32 \%2, 0} & $\rightarrow$ & \llvminline{i1} \\
\textbf{global vars.} & \llvminline{@var = common global i32 0} & $\rightarrow$ & \llvminline{i32} \\
\end{tabular}\\
\medskip
These objects that have a type are called (somewhat confusingly)\\\alert{LLVM Values}.
\begin{block}{The in-memory representation}
\cppinline{llvm::Value} is the \alert{base class} of almost all interesting LLVM-IR objects!
\end{block}
\end{frame}


\begin{frame}{LLVM-IR}{Hierarchy}
\centering
% !TEX root = ../main.tex
% class-hier: core class hierarchy of LLVM

\begin{tikzpicture}
[
  every node/.style={
    font=\tt\scriptsize,
    text depth=0pt
  },
  extends/.style={
    draw,
    open triangle 90-
  },
  level distance=10mm,
  sibling distance=25mm,
  level 3/.style={
    sibling distance=29mm
  },
  level 4/.style={
    sibling distance=36mm
  },
]

\node{llvm::Value}
  [edge from parent fork down,
   edge from parent/.style={extends}]

  child{ node{\cppinline{llvm::Argument}} }
  child{ node{\cppinline{llvm::BasicBlock}} }
  child{ node{\cppinline{llvm::User}}
  	child{ node{\cppinline{llvm::Constant}}
			child{ node{\cppinline{llvm::ConstantData}} }
			child{ node{\cppinline{llvm::ConstantExpr}} }
			child{ node{\cppinline{llvm::GlobalValue}} 
				child{ node{\cppinline{llvm::GlobalObject}} 
					child{ node{\cppinline{llvm::Function}} }
					child{ node{\cppinline{llvm::GlobalVariable}} }
				}
				child{ node{\cppinline{llvm::GlobalAlias}} }
			}
		}
		child{ node{\cppinline{llvm::Instruction}} }
		child{ node{\cppinline{llvm::Operator}} }
  };
\end{tikzpicture}
\\
%\pause
%\bigskip
%Something is missing...
\end{frame}


\begin{frame}{LLVM-IR}{Static Single Assignment}
LLVM-IR is SSA-based:

\begin{itemize}
\item every register is \alert{statically assigned} exactly \alert{once}
\end{itemize}
\bigskip
Statically means that:

\begin{itemize}
\item inside each function...
\item ...for each register \llvminline{\%foo}...
\item ...there is \alert{only one} statement in the form \llvminline{\%foo = ...}
\end{itemize}
\bigskip
\alert{Static} (compile time) $\neq$ \alert{dynamic} (runtime)
{\footnotesize
\begin{itemize}
\item Single \emph{Dynamic} Assignment:\\\emph{in the execution trace} there is only one assignment to a variable \texttt{x}
\item Single \emph{Static} Assignment:\\\emph{in the code listing} there is only one assignment to a variable \texttt{x}
\begin{itemize}
\scriptsize
\item Assignments \alert{can} be performed multiple times (in a loop for example)
\end{itemize}
\end{itemize}
}
\end{frame}


\begin{frame}{Static Single Assignment}{Examples}
\begin{block}{Scalar SAXPY}
\cinput[\ttfamily\small]{snippet/scalar-saxpy.c}
\end{block}

\begin{block}{Scalar LLVM SAXPY}
\llvminput[\ttfamily\small]{snippet/scalar-saxpy.ll}
\end{block}

Temporary \llvminline{\%1} not reused! \llvminline{\%2} is used for the second
assignment!
\end{frame}


\begin{frame}{Static Single Assignment}{Examples}
\begin{block}{Array SAXPY}
\cinput[\ttfamily\scriptsize]{snippet/array-saxpy.c}
\end{block}

\begin{block}{Array LLVM SAXPY}
\llvminput[\ttfamily\scriptsize]{snippet/array-saxpy.ll}
\end{block}

One assignment for loop counter \llvminline{\%i.0}
\end{frame}

\begin{frame}{Static Single Assignment}{Handling Multiple Assignments}
\begin{block}{Max}
\cinput[\ttfamily\scriptsize]{snippet/max.c}
\end{block}

\begin{block}{LLVM Max -- WRONG}
\llvminput[\ttfamily\scriptsize]{snippet/bad-max.ll}
\end{block}

Why is it \alert{wrong}?
\end{frame}


\begin{frame}{Static Single Assignment}{Use \llvminline{phi} to Avoid Troubles}
The \llvminline{\%2} variable must be statically assigned once!\\
How do we handle conditional assignments then?

\begin{block}{LLVM Max}
\llvminput[\ttfamily\footnotesize]{snippet/good-max.ll}
\end{block}

The \llvminline{phi} instruction is a \emph{conditional move}:

\begin{itemize}
\item it takes $(\textrm{variable}_i, \textrm{label}_i)$ pairs
\item if coming from predecessor identified by $\textrm{label}_i$, its value is $\textrm{variable}_i$
\end{itemize}
\end{frame}


\begin{frame}{Static Single Assignment}{Definition and Uses}
Each SSA variable is assigned only once:

\begin{itemize}
\item variable \alert{definition}
\end{itemize}

\vfill
Each SSA variable can be referenced by multiple instructions:

\begin{itemize}
\item variable \alert{uses}
\end{itemize}

\vfill
Algorithms and technical language abuse of these terms!

\vfill
\emph{
Let \llvminline{\%foo} be a variable. If the definition of \llvminline{\%foo} does not
have side-effects nor uses, the aforementioned \llvminline{\%foo} variable 
can be erased from the CFG without altering program semantics.
}
\end{frame}


\begin{frame}{SSA \& LLVM-IR}{Static Single Assignment}
\large
\begin{block}{\centering Important observation}
\centering
SSA means that\\
\alert{there is always a 1:1 correspondence}\\
\alert{between a register and the instruction that assigns it}.
\end{block}
\bigskip
\begin{block}{\centering Consequence}
\centering
As a result, in LLVM-IR\\
\alert{registers are not separate objects}\\
but \alert{every LLVM Instruction\\is the output ``register'' of itself}.
\end{block}
\end{frame}


\begin{frame}{Static Single Assignment}{Rationale}
Old compilers are not SSA-based:

\begin{itemize}
\item converting non-SSA input into SSA form is expensive
\item cost must be amortized
\end{itemize}

\bigskip
New compilers are SSA-based:

\begin{itemize}
\item SSA easier to work with
\item SSA-based analysis/optimizations are faster
\end{itemize}

%\vfill
%All modern compilers are SSA-based:
%
%\begin{itemize}
%\item exception are old version of the HotSpot Client compiler
%\end{itemize}
\end{frame}



% !TEX root = main.tex

\subsection{The Control Flow Graph}


\begin{frame}{A step back...}
Remember how we described the internal structure of an LLVM-IR module:
\begin{itemize}
\item \cppinline{llvm::Module} is a list of \cppinline{llvm::GlobalValue}s.
	\begin{itemize}
	\item \cppinline{llvm::Function} is a kind of \cppinline{llvm::GlobalValue}.
	\end{itemize}
\item \cppinline{llvm::Function} is a list of \cppinline{llvm::BasicBlock}s.
\item \cppinline{llvm::BasicBlock} is a list of \cppinline{llvm::Instruction}s.
\end{itemize}
\bigskip
Functions and basic blocks act like containers:

\begin{itemize}
\item STL-like accessors: \cppinline{front()}, \cppinline{back()},
      \cppinline{size()}, \ldots
\item STL-like iterators: \cppinline{begin()}, \cppinline{end()}
\end{itemize}

\vfill
Each contained element is aware of its container:
\begin{itemize}
\item \cppinline{getParent()}
\end{itemize}
\vfill
Warning for BBs: order of iteration $\neq$ order of execution!
\end{frame}


\begin{frame}{A step back...}
In a \cppinline{llvm::BasicBlock}, the \cppinline{llvm::Instruction}s execute
in the order specified by the list.\\
\begin{itemize}
\item In which order do the \cppinline{llvm::BasicBlock}s execute?
\end{itemize}
\bigskip
The way the basic blocks are executed is implcitly described by
the \alert{branches} in each block.\\
\begin{itemize}
\item These branches describe the \alert{Control Flow Graph} of the function.
\end{itemize}
\end{frame}


\begin{frame}{Control Flow Graph}
LLVM automatically maintains a simple API for operating on the CFG:

\begin{itemize}
\item no need to run passes
\item no need to search the branch instructions in each basic block
\end{itemize}
\bigskip
Every CFG has an \alert{entry} basic block:

\begin{itemize}
\item the \alert{first} executed basic block
\item it is the \alert{root/source} of the graph
\item get it with \cppinline{llvm::Function::getEntryBlock()}
\end{itemize}

\end{frame}


\begin{frame}{Control Flow Graph}{Walking}
At the end of a basic blocks there's always a \alert{terminator} instruction:
\begin{itemize}
\item \llvminline{ret}, \llvminline{br}, \llvminline{switch}, \llvminline{unreachable}, \ldots
\end{itemize}

\bigskip
More than one \alert{exit} block can be present in a function:

\begin{itemize}
\item they are the \alert{leaves/sinks} of the graph
\item their terminator instructions are always \llvminline{ret}s
\begin{enumerate}
\item \cppinline{llvm::BasicBlock::getTerminator()}
\item check the opcode of the terminator
\end{enumerate}
\end{itemize}
\end{frame}


\begin{frame}{Side Note}{Casting Framework}
For performance reasons, a custom casting framework is used:

\begin{itemize}
\item you cannot use \cppinline{static\_cast} and \cppinline{dynamic\_cast} with
      types/classes provided by LLVM
\end{itemize}

\begin{block}{LLVM Casting Functions}
\centering
\medskip
\begin{tabular}{rl}

Static cast of \cppinline{Y*} to \cppinline{X}  &
\cppinline{X *llvm::cast<X>(Y *)}                \\

Dynamic cast of \cppinline{Y*} to \cppinline{X}  &
\cppinline{X *llvm::dyn\_cast<X>(Y *)}            \\

Is \cppinline{Y*} an instance of \cppinline{X}?  &
\cppinline{bool llvm::isa<X>(Y *)} \\

\end{tabular}
\smallskip
\end{block}

Example:

\begin{itemize}
\item is \cppinline{BB} a sink?\\
      \cppinline{llvm::isa<llvm::ReturnInst>(BB.getTerminator())}
\end{itemize}
\end{frame}


\begin{frame}{Control Flow Graph}{Basic Blocks}
Every basic block \cppinline{BB} has one or more\footnote{see include/llvm/IR/CFG.h}:

\begin{description}[predecessors]
\item[predecessors] from \cppinline{pred\_begin(BB)} to
      \cppinline{pred\_end(BB)}
\item[successors] from \cppinline{succ\_begin(BB)} to
      \cppinline{succ\_end(BB)}
\end{description}

\vfill
Other convenience methods available in \cppinline{llvm::BasicBlock}:

\begin{itemize}
\item useful getters
\begin{itemize}
\item \cppinline{BasicBlock *getUniquePredecessor()}
\item \ldots
\end{itemize}
\item moving a basic block
\begin{itemize}
\item      \cppinline{moveBefore(llvm::BasicBlock *)}
\item      \cppinline{moveAfter(llvm::BasicBlock *)}
\end{itemize}
\item split a basic block:
\begin{itemize}
\item      \cppinline{splitBasicBlock(llvm::BasicBlock::iterator)}
\end{itemize}
\item \ldots
\end{itemize}
\end{frame}


\begin{frame}{Control Flow Graph}{Instructions}
The \cppinline{llvm::Instruction} class defines common operations: \\
\medskip
\begin{itemize}
\item getting an operand
\begin{itemize}
\item \cppinline{getOperand(unsigned)}
\end{itemize}
\end{itemize}
\vfill
Subclasses provide specialized accessors: \\
\medskip
\begin{itemize}
\item the \llvminline{load} instruction takes as operand the pointer to the memory to be loaded:
\begin{itemize}
\item      \cppinline{llvm::LoadInst::getPointerOperand()}
\end{itemize}
\end{itemize}
\end{frame}


\begin{frame}{Instructions}{Creating New Instructions}
Instructions are created using:

\begin{itemize}
\item constructors
\begin{itemize}
\item \cppinline{llvm::LoadInst::LoadInst(...)}
\end{itemize}
\item factory methods
\begin{itemize}
\item \cppinline{llvm::GetElementPtrInst::Create(...)}
\end{itemize}
\item the helper class \cppinline{llvm::IRBuilder}
\begin{itemize}
\item \cppinline{llvm::IRBuilder<> builder(insPoint);}\\
\cppinline{builder.CreateAdd(...);}
\end{itemize}
\end{itemize}
\vfill
\alert{Interface is not homogeneous!}\\
Some instructions support all methods, others support only one.
\end{frame}


\begin{frame}{Instructions}{Inserting New Instructions}
\vfill
Instructions can be inserted:
\vfill
\begin{itemize}
\item automatically by \cppinline{IRBuilder}
\begin{itemize}
\item insertion point is given at \cppinline{IRBuilder} instantiation
\end{itemize}
\bigskip
\item manually by appending to a basic block
\item manually by inserting after/before another instruction
\end{itemize}
\vfill
\end{frame}


\begin{frame}{From Control Flow to Data Flow}{Definitions and Uses}
In LLVM, the data flow generated by the various instructions is represented by a simple hierarchy:

\begin{description}[valueMMM]
\item[value] something that can be used: \cppinline{llvm::Value}
\item[user] something that can use: \cppinline{llvm::User}
\item[use] the link between the \alert{value} and the \alert{user}: \cppinline{llvm::Use}
\end{description}
\medskip
A value is a \alert{definition}:

\begin{itemize}
\item Visiting where a definition is used:
\begin{itemize}
\item \cppinline{llvm::Value::use\_begin()}, \cppinline{llvm::Value::use\_end()}
\end{itemize}
\end{itemize}
\medskip
An user accesses \alert{definitions}:

\begin{itemize}
\item Visiting the definitions that are used:
\begin{itemize}
\item \cppinline{llvm::User::op\_begin()}, \cppinline{llvm::User::op\_end()}
\end{itemize}
\end{itemize}
\medskip

\end{frame}


\begin{frame}{From Control Flow to Data Flow}{Instructions are Values}
\vfill
\begin{itemize}
\item \cppinline{llvm::Value} inherits from \cppinline{llvm::User}
\item \cppinline{llvm::Instruction} inherits from \cppinline{llvm::Value}
\begin{itemize}
\normalsize
\item[$\Rightarrow$] The value produced by the instruction is\\the \alert{instruction itself}!
\end{itemize}
\end{itemize}

\begin{block}{Example}
\begin{center}
\llvminline{\%6 = load i32, i32* \%1, align 4}\\
\medskip
The \llvminline{load} is described by an instance of \cppinline{llvm::Instruction}. \\
That instance also represents the \llvminline{\%6} variable. \\
\end{center}
\end{block}

\begin{center}
Not all instances of \cppinline{llvm::Value} are also \cppinline{llvm::Instruction}s!\\
\smallskip
{\small i.e. function arguments}

\end{center}
\vfill
\end{frame}


\begin{frame}{From Control Flow to Data Flow}{Value Typing}
Every \cppinline{llvm::Value} is typed:

\begin{itemize}
\item use \cppinline{llvm::Value::getType()} to get the type
\end{itemize}

\vfill
Since every instruction is a value:

\begin{itemize}
\item instructions are typed
\end{itemize}

\vfill
\begin{block}{Example}
\begin{center}
\llvminline{\%6 = load i32, i32* \%1, align 4}\\
\medskip
The type of the \llvminline{\%6} variable is the type of the return value of the \llvminline{load} instruction, \llvminline{i32}\\
\end{center}
\end{block}
\end{frame}


%\begin{frame}{From Control Flow to Data Flow}{Instructions are like functions}
%\vfill
%\vfill
%\begin{columns}[t]
%\column{.45\textwidth}
%Functions:
%
%\begin{itemize}
%\item used by call sites
%\item uses formal parameters
%\end{itemize}
%
%\column{.45\textwidth}
%Instructions:
%
%\begin{itemize}
%\item define an SSA value
%\item uses operands
%\end{itemize}
%\end{columns}
%
%\vfill
%\vfill
%\end{frame}





% !TEX root = main.tex

\section{Algorithm design}


\begin{frame}{Classical Algorithm Design}
In algorithm design, a good approach is the following:
\begin{enumerate}
\item study the problem
\item make some example
\item identify the \alert{common case}
\item derive the algorithm for the common case
\item add handling for \alert{corner cases}
\item improve performancing \alert{optimizing the common case}
\end{enumerate}

\vfill
Weakness of the approach:
\begin{itemize}
\item \alert{corner cases} -- a \emph{correct} algorithm \textbf{must} consider \emph{all the corner cases}!
\end{itemize}
\end{frame}


\begin{frame}{Compiler Algorithm Design}{Be Conservative}
\begin{center}
Corner cases are difficult to handle, but they cannot be ignored\\
\smallskip
{\small Compiler algorithms must be \alert{proven} to preserve\\
program semantic \alert{at all times}}\\
\bigskip
As an aid, a \emph{standard methodology} is employed.\\
\bigskip
Compiler algorithms are built combining \alert{three} kinds of passes:\\
\medskip
Analysis, Optimization, Normalization\\
\bigskip
\pause
We now consider a simple example: \emph{loop hoisting}.
\end{center}
\end{frame}


\begin{frame}{Loop Hoisting}
It is a transformation that:
\begin{itemize}
\item looks for statements (inside a loop) not depending on the loop state
\item move them outside the loop body
\end{itemize}

\begin{columns}[t]
\column{.45\textwidth}
\begin{block}{Loop Hoisting -- Before}
\centering
\cinput{snippet/loop-hoisting-before.c}
\end{block}

\column{.45\textwidth}
\begin{block}{Loop Hoisting -- After}
\centering
\cinput{snippet/loop-hoisting-after.c}
\end{block}
\end{columns}
\end{frame}


\begin{frame}{Loop Hoisting}{Focus on the Transformation}
\begin{block}{The general idea:}
\begin{itemize}
\item move ``good'' statement outside of the loop
\end{itemize}
\end{block}

This \alert{pass} modifies the code, thus it is an \alert{optimization pass}.

It needs to know:
\begin{itemize}
\item which pieces of code are loops
\item which statements are ``good'' statements
\end{itemize}

This information is computed by the the \alert{analysis} passes:

\begin{itemize}
\item detecting loops in the program
\item detecting loop-independent statements
\end{itemize}

The loop hoisting pass declares which analyses it needs:

\begin{itemize}
\item pipeline automatically built: \alert{analysis $\rightarrow$ optimization}
\end{itemize}
\end{frame}


\begin{frame}{Loop Hoisting}{Proving Program Semantic Preservation}
The \alert{proof} is trivial:

\begin{itemize}
\item the transformation shall preserve program semantics
\item the analyses shall be correct
\end{itemize}

Analysis passes are usually built starting from other analyses already
implemented inside the compiler, or are already present in LLVM
\begin{itemize}
\item often, no proof is necessary for the analyses
\end{itemize}

\vfill
\begin{center}
\alert{However...}\\
\smallskip
You also have to prove that the combination of analysis + transformation is correct!\\
\smallskip
"Beware of bugs in the above code;\\I have only proved it correct, not tried it."\\--- Knuth
\end{center}
\end{frame}


\begin{frame}{Loop Hoisting}{More Loops}
\begin{center} 
We have spoken about loops, but which kind of loop?\\
\medskip
\cinline{do-while}? \cinline{while}? \cinline{for}?\\

\bigskip
Almost all loops are different forms of the \alert{same exact thing}\\
$\Downarrow$\\
We can convert a lot of loops to a loop of another kind!\\

\bigskip
To account for the various kinds of loops, we choose a \alert{normal}
kind of loop, and then we write a \alert{normalization} pass.\\
\smallskip
{\footnotesize
Usually, \cinline{do-while} loops are chosen to be the \emph{normal} loops.\\
\vspace{-1mm}
Sometimes, normalization is also called \alert{canonicalization}.}\\

\bigskip
The more loops we recognize, the higher the potential\\
\alert{optimization impact}!
\end{center}
\end{frame}


\begin{frame}{Compiler Algorithm Design}{A methodology}
You have to:

\begin{enumerate}
\item analyze the problem
\item make some examples
\item detect the common case
\item determine the \alert{input conditions}
\item determine which \alert{analyses} you need
\item design the \alert{optimization} pass
\item proof its \alert{correctness}
\item improve algorithm perfomance on the common case
\item improve the effectiveness of the algorithm by adding
      \alert{normalization passes}
\end{enumerate}
\end{frame}


\begin{frame}{Compiler Algorithm Design}{Ignore corner cases!}
\begin{center}
Something is missing...\\
\bigskip
{\Large Corner Cases!}\\
\bigskip
Why?\\
\end{center}
\begin{enumerate}
\item It makes no sense to optimize code that is seldom executed
\item Your optimization will be based on \alert{properties of the code that are true only in the common case you are considering}
\begin{itemize}
\item If the code does not fit the common case, it shall stay as-is
\item Otherwise you \alert{risk breaking program semantics}!
\end{itemize}
\end{enumerate}
\end{frame}


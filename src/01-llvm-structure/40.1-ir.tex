% !TEX root = main.tex

\subsection{The LLVM-IR language}


\begin{frame}{How passes work}
\centering
A \alert{pass} is a \alert{subroutine} that \\programmatically
transforms a piece of code.\\
\bigskip
The code to be transformed \\is represented in a language 
called \alert{LLVM-IR}.\\
\bigskip
The code of the pass operates on the LLVM-IR \\using a set of
\alert{object-oriented APIs}.
\vfill
Let's talk about LLVM-IR~\cite{LOCAL:www/llvmLanguageRef}, 
first by looking at its \alert{human-readable} form.
\end{frame}


\begin{frame}{LLVM-IR}{Example: factorial}
\begin{center}
\begin{varwidth}{20cm}
\llvminput[\ttfamily\fontsize{7pt}{5pt}\selectfont]{snippet/fact.ll}
\end{varwidth}
\end{center}
\end{frame}


\begin{frame}{LLVM-IR}{Top Level}
From top to bottom:
\begin{itemize}
\item The topmost object of a LLVM-IR program is the \alert{Module}.
\item \alert{Modules} contain a list of \alert{Globals}.
	\begin{itemize}
	\item {Globals} can be either \alert{Functions} or \alert{Global Variables}.
	\item A global can be a \alert{Forward declaration}.
	\end{itemize}
\item \alert{Functions} contain a list of \alert{Basic Blocks} + \alert{Arguments}.
\item \alert{Basic Blocks} are a list of \alert{Instructions}.
\end{itemize}
\begin{block}{The in-memory representation}
All these parts will correspond directly to \alert{C++ objects}.\\
The abundance of lists guarantees low overhead and scalability to very large programs.
\end{block}
\end{frame}


\begin{frame}{LLVM-IR}{Instructions}
LLVM-IR looks a lot like a RISC assembly language:\\

\begin{itemize}
\item Few instructions, all perfectly orthogonal
	\begin{itemize}
	\item There are infinite registers
	\item There are no special-purpose registers
	\item No implicit flags register
	\end{itemize}
\item Basic block boundaries are denoted by \alert{labels}
\item Only \llvminline{load} and \llvminline{store} access memory
\end{itemize}

\vfill
There are also a few CISC-like \alert{high level instructions}:

\begin{itemize}
\item Reserve memory on the stack -- \llvminline{alloca}
\item Function call -- \llvminline{call}
	\begin{itemize}
	\item The calling convention is abstracted away
	\item There is an implicit call stack
	\end{itemize}
\item Pointer arithmetics -- \llvminline{getelementptr}
\item \ldots
\end{itemize}
\end{frame}


\begin{frame}{LLVM-IR}{Values \& Types}
LLVM-IR is \alert{strongly typed}:

\begin{itemize}
\item e.g. you cannot assign a floating point value to an integer register
without an explicit cast
\end{itemize}
\medskip
\alert{Almost everything} is \alert{typed}:\\
\medskip
\begin{tabular}{>{\RaggedLeft\arraybackslash}p{5.5em}lcl}
\textbf{functions} & \llvminline{@fact} & $\rightarrow$ & \llvminline{i32 (i32)} \\
\textbf{registers} & \llvminline{\%3 = icmp eq i32 \%2, 0} & $\rightarrow$ & \llvminline{i1} \\
\textbf{globals var.} & \llvminline{@var = common global i32 0} & $\rightarrow$ & \llvminline{i32} \\
\end{tabular}\\
\medskip
These objects that have a type are called (somewhat confusingly)\\\alert{LLVM Values}.
\begin{block}{The in-memory representation}
\cppinline{llvm::Value} is the \alert{base class} of almost all interesting LLVM-IR objects!
\end{block}
\end{frame}


\begin{frame}{LLVM-IR}{Hierarchy}
\centering
% !TEX root = ../main.tex
% class-hier: core class hierarchy of LLVM

\begin{tikzpicture}
[
  every node/.style={
    font=\tt\scriptsize,
    text depth=0pt
  },
  extends/.style={
    draw,
    open triangle 90-
  },
  level distance=10mm,
  sibling distance=25mm,
  level 3/.style={
    sibling distance=29mm
  },
  level 4/.style={
    sibling distance=36mm
  },
]

\node{llvm::Value}
  [edge from parent fork down,
   edge from parent/.style={extends}]

  child{ node{\cppinline{llvm::Argument}} }
  child{ node{\cppinline{llvm::BasicBlock}} }
  child{ node{\cppinline{llvm::User}}
  	child{ node{\cppinline{llvm::Constant}}
			child{ node{\cppinline{llvm::ConstantData}} }
			child{ node{\cppinline{llvm::ConstantExpr}} }
			child{ node{\cppinline{llvm::GlobalValue}} 
				child{ node{\cppinline{llvm::GlobalObject}} 
					child{ node{\cppinline{llvm::Function}} }
					child{ node{\cppinline{llvm::GlobalVariable}} }
				}
				child{ node{\cppinline{llvm::GlobalAlias}} }
			}
		}
		child{ node{\cppinline{llvm::Instruction}} }
		child{ node{\cppinline{llvm::Operator}} }
  };
\end{tikzpicture}
\\
\pause
\bigskip
Something is missing...
\end{frame}


\begin{frame}{LLVM-IR}{Static Single Assignment}
LLVM-IR is SSA-based:

\begin{itemize}
\item every register is \alert{statically assigned} exactly \alert{once}
\end{itemize}
\bigskip
Statically means that:

\begin{itemize}
\item inside each function...
\item ...for each variable \llvminline{\%foo}...
\item ...there is \alert{only one} statement in the form \llvminline{\%foo = ...}
\end{itemize}
\bigskip
\alert{Static} (compile time) $\neq$ \alert{dynamic} (runtime)
{\footnotesize
\begin{itemize}
\item Single \emph{Dynamic} Assignment:\\\emph{in the execution trace} there is only one assignment to a variable \texttt{x}
\item Single \emph{Static} Assignment:\\\emph{in the code listing} there is only one assignment to a variable \texttt{x}
\begin{itemize}
\scriptsize
\item Assignments \alert{can} be performed multiple times (in a loop for example)
\end{itemize}
\end{itemize}
}
\end{frame}


\begin{frame}{Static Single Assignment}{Examples}
\begin{block}{Scalar SAXPY}
\cinput[\ttfamily\small]{snippet/scalar-saxpy.c}
\end{block}

\begin{block}{Scalar LLVM SAXPY}
\llvminput[\ttfamily\small]{snippet/scalar-saxpy.ll}
\end{block}

Temporary \llvminline{\%1} not reused! \llvminline{\%2} is used for the second
assignment!
\end{frame}


\begin{frame}{Static Single Assignment}{Examples}
\begin{block}{Array SAXPY}
\cinput[\ttfamily\scriptsize]{snippet/array-saxpy.c}
\end{block}

\begin{block}{Array LLVM SAXPY}
\llvminput[\ttfamily\scriptsize]{snippet/array-saxpy.ll}
\end{block}

One assignment for loop counter \llvminline{\%i.0}
\end{frame}

\begin{frame}{Static Single Assignment}{Handling Multiple Assignments}
\begin{block}{Max}
\cinput[\ttfamily\scriptsize]{snippet/max.c}
\end{block}

\begin{block}{LLVM Max -- WRONG}
\llvminput[\ttfamily\scriptsize]{snippet/bad-max.ll}
\end{block}

Why is it \alert{wrong}?
\end{frame}


\begin{frame}{Static Single Assignment}{Use \llvminline{phi} to Avoid Troubles}
The \llvminline{\%2} variable must be statically assigned once!\\
How do we handle conditional assignments then?

\begin{block}{LLVM Max}
\llvminput[\ttfamily\footnotesize]{snippet/good-max.ll}
\end{block}

The \llvminline{phi} instruction is a \emph{conditional move}:

\begin{itemize}
\item it takes $(\textrm{variable}_i, \textrm{label}_i)$ pairs
\item if coming from predecessor identified by $\textrm{label}_i$, its value is $\textrm{variable}_i$
\end{itemize}
\end{frame}


\begin{frame}{Static Single Assignment}{Definition and Uses}
Each SSA variable is assigned only once:

\begin{itemize}
\item variable \alert{definition}
\end{itemize}

\vfill
Each SSA variable can be referenced by multiple instructions:

\begin{itemize}
\item variable \alert{uses}
\end{itemize}

\vfill
Algorithms and technical language abuse of these terms!

\vfill
\emph{
Let \llvminline{\%foo} be a variable. If the definition of \llvminline{\%foo} does not
have side-effects nor uses, the aforementioned \llvminline{\%foo} variable 
can be erased from the CFG without altering program semantics.
}
\end{frame}


\begin{frame}{SSA \& LLVM-IR}{Static Single Assignment}
\large
\begin{block}{\centering Important observation}
\centering
SSA means that\\
\alert{there is always a 1:1 correspondence}\\
\alert{between a register and the instruction that assigns it}.
\end{block}
\bigskip
\begin{block}{\centering Consequence}
\centering
As a result, in LLVM-IR\\
\alert{registers are not separate objects}\\
but \alert{every LLVM Instruction\\is the output ``register'' of itself}.
\end{block}
\end{frame}


\begin{frame}{Static Single Assignment}{Rationale}
Old compilers are not SSA-based:

\begin{itemize}
\item converting non-SSA input into SSA form is expensive
\item cost must be amortized
\end{itemize}

\vfill
New compilers are SSA-based:

\begin{itemize}
\item SSA easier to work with
\item SSA-based analysis/optimizations are faster
\end{itemize}

%\vfill
%All modern compilers are SSA-based:
%
%\begin{itemize}
%\item exception are old version of the HotSpot Client compiler
%\end{itemize}
\end{frame}



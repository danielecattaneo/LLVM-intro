\documentclass[10pt,usenames,dvipsnames]{beamer}
\usepackage[english]{babel}
\usepackage{euler}

\usetheme{boxes}
%\useoutertheme{essential}
\usecolortheme{seagull}
\usefonttheme{professionalfonts}
\usefonttheme{structurebold}
\setbeamertemplate{navigation symbols}{}
\setbeamertemplate{frametitle}
{
	\begin{centering}
		\vspace{1.5em}
		\LARGE
    \insertframetitle
    \par
    \vspace{0.5em}
  \end{centering}
}
\renewcommand{\thefootnote}{\textsf{\fnsymbol{footnote}}}

% footnote refs per frame
\AtBeginEnvironment{frame}{\setcounter{footnote}{0}}

\usepackage[no-math]{fontspec}

\usepackage{tikz}
\usepackage{listings}
\usepackage{relsize}
\usepackage{array}
\usepackage{booktabs}
\usepackage{ragged2e}
\usepackage{varwidth}
\usepackage{multicol}

\makeatletter
\newlength{\negph@width}%
\newcommand{\negphantom}[1]{%
\settowidth{\negph@width}{#1}%
\hspace{-\negph@width}%
}
\makeatother

\newcommand{\aligntext}[2]{%
#2\negphantom{#2}\hphantom{#1}%
}

\setmainfont{texgyretermes}[
  Path=../fonts/,
  Extension=.otf,
  UprightFont=*-regular,
  BoldFont=*-bold,
  ItalicFont=*-italic,
  BoldItalicFont=*-bolditalic]
\setsansfont{texgyreheros}[
  Path=../fonts/,
  Extension=.otf,
  UprightFont=*-regular,
  BoldFont=*-bold,
  ItalicFont=*-italic,
  BoldItalicFont=*-bolditalic]
\setmonofont{RecMono-Casual}[
  Path=../fonts/,
  Extension=.ttf,
  BoldFont=*Bold,
  ItalicFont=*Italic,
  BoldItalicFont=*BoldItalic]

\usetikzlibrary{arrows}
\usetikzlibrary{backgrounds}
\usetikzlibrary{chains}
\usetikzlibrary{fit}
\usetikzlibrary{positioning}
\usetikzlibrary{scopes}
\usetikzlibrary{trees}
\usetikzlibrary{automata}
\usetikzlibrary{positioning}
\usetikzlibrary{shapes.multipart}

\lstset{basicstyle=\ttfamily}
\newcommand{\cinput}[2][\ttfamily]{\lstinputlisting[language=C,basicstyle=#1]{#2}}
\newcommand{\cinline}[2][\ttfamily]{\lstinline[language=C,basicstyle=#1]!#2!}
\newcommand{\cppinput}[2][\ttfamily]{\lstinputlisting[language=C++,basicstyle=#1]{#2}}
\newcommand{\cppinline}[2][\ttfamily]{\lstinline[language=C++,basicstyle=#1]!#2!}
\newcommand{\llvminput}[2][\ttfamily]{\lstinputlisting[language=LLVM,basicstyle=#1]{#2}}
\newcommand{\llvminline}[2][\ttfamily]{\lstinline[language=LLVM,basicstyle=#1]!#2!}
\lstdefinelanguage{LLVM}%
  {morekeywords={define,declare,global,constant,internal,external,private,%
      linkonce,linkonce_odr,weak,weak_odr,appending,common,extern_weak,%
      thread_local,dllimport,dllexport,hidden,protected,default,except,deplibs,%
      volatile,fastcc,coldcc,cc,ccc,x86_stdcallcc,x86_fastcallcc,ptx_kernel,%
      ptx_device,signext,zeroext,inreg,sret,nounwind,noreturn,nocapture,byval,%
      nest,readnone,readonly,noalias,uwtable,inlinehint,noinline,alwaysinline,%
      optsize,ssp,sspreq,noredzone,noimplicitfloat,naked,alignstack,module,asm,%
      align,tail,to,addrspace,section,alias,sideeffect,c,gc,target,datalayout,%
      triple,blockaddress},%
  morekeywords=[2]{add,fadd,sub,fsub,mul,fmul,sdiv,udiv,fdiv,srem,urem,frem,%
     and,or,xor,icmp,fcmp,eq,ne,ugt,uge,ult,ule,sgt,sge,slt,sle,oeq,ogt,oge,%
     olt,ole,one,ord,ueq,ugt,uge,ult,ule,une,uno,nuw,nsw,exact,inbounds,phi,%
     call,select,shl,lshr,ashr,va_arg,trunc,zext,sext,fptrunc,fpext,fptoui,%
     fptosi,uitofp,sitofp,ptrtoint,inttoptr,bitcast,ret,br,indirectbr,switch,%
     invoke,unwind,unreachable,malloc,alloca,free,load,store,getelementptr,%
     extractelement,insertelement,shufflevector,extractvalue,insertvalue},%
  sensitive=t,%
  morestring=[b]",%
  morecomment=[l];%
  }[keywords,comments,strings]
  
% Create Color definition From Template: 
% #1 template name, 
% #2 foreground color name
% #3 background color name
\newcommand{\ccft}[3]{
\usebeamercolor{#1}
\definecolor{#2}{named}{fg}
\definecolor{#3}{named}{bg}
}
\ccft{palette primary}{ThemePriFg}{ThemePriBg}
\ccft{palette secondary}{ThemeSecFg}{ThemeSecBg}
\ccft{palette tertiary}{ThemeTerFg}{ThemeTerBg}
\ccft{palette quaternary}{ThemeQuaFg}{ThemeQuaBg}


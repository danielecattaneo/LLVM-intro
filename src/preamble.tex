\documentclass[10pt,usenames,dvipsnames]{beamer}
\usepackage[english]{babel}
\usepackage{euler}

\usetheme{boxes}
%\useoutertheme{essential}
\usecolortheme{seagull}
\usefonttheme{professionalfonts}
\usefonttheme{structurebold}
\setbeamertemplate{navigation symbols}{}
\setbeamertemplate{frametitle}
{
	\begin{centering}
		\vspace{1.5em}
		\LARGE
    \insertframetitle
    \par
    %\vspace{0.5em}
  \end{centering}
}
\setbeamerfont{title}{size=\huge}
\setbeamerfont{subtitle}{size=\Large}
\renewcommand{\thefootnote}{\textsf{\fnsymbol{footnote}}}

% footnote refs per frame
\AtBeginEnvironment{frame}{\setcounter{footnote}{0}}

\usepackage[no-math]{fontspec}

\usepackage{tikz}
\usepackage{listings}
\usepackage{relsize}
\usepackage{array}
\usepackage{booktabs}
\usepackage{ragged2e}
\usepackage{varwidth}
\usepackage{multicol}

\makeatletter
\newlength{\negph@width}%
\newcommand{\negphantom}[1]{%
\settowidth{\negph@width}{#1}%
\hspace{-\negph@width}%
}
\makeatother

\newcommand{\aligntext}[2]{%
#2\negphantom{#2}\hphantom{#1}%
}

\setmainfont{texgyretermes}[
  Path=../fonts/,
  Extension=.otf,
  UprightFont=*-regular,
  BoldFont=*-bold,
  ItalicFont=*-italic,
  BoldItalicFont=*-bolditalic]
\setsansfont{texgyreheros}[
  Path=../fonts/,
  Extension=.otf,
  UprightFont=*-regular,
  BoldFont=*-bold,
  ItalicFont=*-italic,
  BoldItalicFont=*-bolditalic]
\setmonofont{RecMono-Casual}[
  Path=../fonts/,
  Extension=.ttf,
  BoldFont=*Bold,
  ItalicFont=*Italic,
  BoldItalicFont=*BoldItalic]

\usetikzlibrary{arrows}
\usetikzlibrary{backgrounds}
\usetikzlibrary{chains}
\usetikzlibrary{fit}
\usetikzlibrary{positioning}
\usetikzlibrary{scopes}
\usetikzlibrary{trees}
\usetikzlibrary{automata}
\usetikzlibrary{positioning}
\usetikzlibrary{shapes.multipart}

\lstset{basicstyle=\ttfamily}
\newcommand{\cinput}[2][\ttfamily]{\lstinputlisting[language=C,basicstyle=#1]{#2}}
\newcommand{\cinline}[2][\ttfamily]{\lstinline[language=C,basicstyle=#1]!#2!}
\newcommand{\cppinput}[2][\ttfamily]{\lstinputlisting[language=C++,basicstyle=#1]{#2}}
\newcommand{\cppinline}[2][\ttfamily]{\lstinline[language=C++,basicstyle=#1]!#2!}
\newcommand{\llvminput}[2][\ttfamily]{\lstinputlisting[language=LLVM,basicstyle=#1]{#2}}
\newcommand{\llvminline}[2][\ttfamily]{\lstinline[language=LLVM,basicstyle=#1]!#2!}
\newcommand{\asminput}[2][\ttfamily]{\lstinputlisting[language=x86gas,basicstyle=#1]{#2}}
\newcommand{\asminline}[2][\ttfamily]{\lstinline[language=x86gas,basicstyle=#1]!#2!}
\lstdefinelanguage{LLVM}%
  {morekeywords={define,declare,global,constant,internal,external,private,%
      linkonce,linkonce_odr,weak,weak_odr,appending,common,extern_weak,%
      thread_local,dllimport,dllexport,hidden,protected,default,except,deplibs,%
      volatile,fastcc,coldcc,cc,ccc,x86_stdcallcc,x86_fastcallcc,ptx_kernel,%
      ptx_device,signext,zeroext,inreg,sret,nounwind,noreturn,nocapture,byval,%
      nest,readnone,readonly,noalias,uwtable,inlinehint,noinline,alwaysinline,%
      optsize,ssp,sspreq,noredzone,noimplicitfloat,naked,alignstack,module,asm,%
      align,tail,to,addrspace,section,alias,sideeffect,c,gc,target,datalayout,%
      triple,blockaddress},%
  morekeywords=[2]{add,fadd,sub,fsub,mul,fmul,sdiv,udiv,fdiv,srem,urem,frem,%
     and,or,xor,icmp,fcmp,eq,ne,ugt,uge,ult,ule,sgt,sge,slt,sle,oeq,ogt,oge,%
     olt,ole,one,ord,ueq,ugt,uge,ult,ule,une,uno,nuw,nsw,exact,inbounds,phi,%
     call,select,shl,lshr,ashr,va_arg,trunc,zext,sext,fptrunc,fpext,fptoui,%
     fptosi,uitofp,sitofp,ptrtoint,inttoptr,bitcast,ret,br,indirectbr,switch,%
     invoke,unwind,unreachable,malloc,alloca,free,load,store,getelementptr,%
     extractelement,insertelement,shufflevector,extractvalue,insertvalue},%
  sensitive=t,%
  morestring=[b]",%
  morecomment=[l];%
  }[keywords,comments,strings]
\lstdefinelanguage{x86gas}%
  {morekeywords={%
.abort, .ABORT, .align, .ascii, .asciz, .balign, .bss, .bundle, .byte, .cfi, %
.comm, .cstring, .data, .def, .desc, .dim, .double, .eject, .else, .elseif, %
.end, .endef, .endfunc, .endif, .equ, .equiv, .eqv, .err, .error, .exitm, %
.extern, .fail, .file, .fill, .float, .func, .global, .globl, .gnu, .hidden, %
.hword, .ident, .if, .incbin, .include, .int, .internal, .irp, .irpc, .lcomm, %
.lflags, .line, .ln, .linkonce, .list, .loc, .local, .long, .macro, .mri, %
.noaltmacro, .nolist, .octa, .offset, .org, .p2align, .popsection, .previous, %
.print, .protected, .psize, .purgem, .pushsection, .quad, .rept, .sbttl, .scl, %
.section, .set, .short, .single, .size, .skip, .sleb128, .space, .stabd, %
.stabn, .stabs, .string, .struct, .subsection, .symver, .tag, .text, .title, %
.type, .uleb128, .val, .version, .vtable, 	.vtable_entry, 	.vtable_inherit, %
.warning, .weak, .weakref, .word, .zero
},%
  morekeywords=[2]{%
aaa, adcb, adcl, add, addb, addl, addps, addw, addq, adds, addss, addsd, and, %
andb, andw, andl, andq, bswap, call, callq, cld, cltd, cltq, cmova, cmovae, %
cmovb, cmovbe, cmovg, cmovge, cmovl, cmovle, cmovna, cmovnae, cmovnb, cmovnbe, %
cmovne, cmovng, cmovnge, cmovnl, cmovnle, cmovns, cmovnz, cmovs, cmovz, cmp, %
cmpb, cmpeqps, cmpl, cmpq, cmpsl, cmpxchg, cpuid, cqto, cvtps2dq, cvtdq2ps, %
cvtsd2ss, cvtsi2s, cvtsi2ss, cvtsi2ssq, cvtsi2sd, cvtsi2sdq, cvtss2sd, cvttsd, %
cvttsd2si, cvttsd2siq, cvtts, cvttss, cvttss2si, cvttss2siq, das, dec, decb, %
decw, decl, decq, divw, divl, divq, divs, divss, divsd, fabs, fadd, fbld, %
fbstp, fchs, fcmovb, fcomi, fcoms, fcos, fdivp, fdivr, fdivrp, fidivs, filds, %
fimul, finit, fists, fld1, fldcw, fldl, fldl2t, fldl2e, fldlenv, fldlcw, %
fldlg2, fldln2, fldpi, flds, fldz, fmul, fmulp, fprem1, frame, frndint, %
frstor, fsave, fsin, fsincos, fsqrt, fst, fstcw, fstl, fstps, fsts, fstsw, %
fstenv, fstpl, fstsw, fsubr, fxch, fyl2x, idivl, idivq, imul, imulb, imull, %
imulw, imulq, inc, incb, incw, incl, incq, int, ja, jb, jc, jcxz, je, jg, jge, %
jl, jmp, jne, jns, jnz, jo, jp, js, jz, lea, leab, leave, leawl, leaq, leal, %
lodsb, lodsl, loop, maxps, maxs, maxss, maxsd, mins, minss, minssd, mov, %
movabsq, movapd, movaps, movqps, movb, movdqa, movdqu, movl, movlpd, movq, %
movupd, movups, movsb, movsbw, movsbl, movsbq, movsd, movsl, movslq, movsw, %
movswl, movswq, movsx, movss, movsd, movw, movz, movzbw, movzbl, movzbq, %
movzwl, movzwq, movzx, msg, mull, mulpd, muls, mulss, mulsd, mulq, neg, negb, %
negw, negl, negq, nop, not, notb, notw, notl, notq, or, orb, orw, orl, orq, %
paddd, pcmpeqw, pop, popfl, popl, popq, popw, push, pushfl, pushl, pushq, %
pushw, rep, repe, repne, ret, retq, rspreg, sahf, sal, salb, salw, sall, salq, %
sar, sarb, sarw, sarl, sarq, sbbb, sbbl, shr, shrb, shrw, shrl, shrq, sqrtps, %
sqrts, sqrtss, sqrtsd, std, stosb, sub, subb, subl, subq, subw, subs, subss, %
subsd, syscall, test, testb, testq, ucomis, ucomisd, ucomiss, xchg, xor, xorb, %
xorps, xorw, xorl, xorq, vxorpd, vaddsd},%
  sensitive=t,%
  morestring=[b]",%
  morecomment=[l]\#%
  }[keywords,comments,strings]
  
% Create Color definition From Template: 
% #1 template name, 
% #2 foreground color name
% #3 background color name
\newcommand{\ccft}[3]{
\usebeamercolor{#1}
\definecolor{#2}{named}{fg}
\definecolor{#3}{named}{bg}
}
\ccft{palette primary}{ThemePriFg}{ThemePriBg}
\ccft{palette secondary}{ThemeSecFg}{ThemeSecBg}
\ccft{palette tertiary}{ThemeTerFg}{ThemeTerBg}
\ccft{palette quaternary}{ThemeQuaFg}{ThemeQuaBg}


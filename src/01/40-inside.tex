% !TEX root = 01.tex

\section{Inside LLVM}


\begin{frame}{Terminology}{Speaking About LLVM IR}
LLVM IR comes with 3 different flavours:

\begin{description}
\item[assembly] human-readable format
\item[bitcode] binary on-disk machine-oriented format
\item[in-memory] binary in-memory format, used during compilation process
\end{description}

All formats have the same expressiveness!

\vfill
File extensions:

\begin{description}
\item[.ll] for assembly files
\item[.bc] for bitcode files
\end{description}
\end{frame}

\begin{frame}{Tools}{C Language Family Front-end}
Writing LLVM assembly by hand is unfeasible:

\begin{itemize}
\item different front-ends available for LLVM
\item use Clang~\cite{LOCAL:www/clang} for the C family
\end{itemize}

The clang driver is compatible with GCC:

\begin{itemize}
\item $\approx$ same command line options
\end{itemize}

\vfill
To generate LLVM IR:

\begin{description}
\item[assembly] \texttt{\smaller clang -emit-llvm -S -o out.ll in.c}
\item[bitcode] \texttt{\smaller  clang -emit-llvm -o out.bc in.c}
\end{description}

It can also generate native code starting from LLVM assembly or LLVM bitcode --
like compiling an assembly file with GCC
\end{frame}

\begin{frame}{Tools}{Playing with LLVM Passes}
LLVM IR can be manipulated using \texttt{\smaller opt}:

\begin{itemize}
\item read an input file
\item run specified LLVM passes on it
\item respecting user-provided order
\end{itemize}

\vfill
Useful passes:

\begin{itemize}
\item print CFG with \texttt{\smaller opt -view-cfg input.ll}
\item print dominator tree with \texttt{\smaller opt -view-dom input.ll}
\item \ldots
\end{itemize}

Pass chaining:

\begin{itemize}
\item run \emph{mem2reg}, then view the CFG with \\
\texttt{\smaller opt -mem2reg -view-cfg input.ll}
\item potentially different results using different option order (\alert{phase/stage ordering})
\end{itemize}
\end{frame}

\begin{frame}{Pass Hierarchy}
LLVM provides a lot of passes:

\begin{itemize}
\item try \texttt{\smaller opt -help}
\end{itemize}

\vfill
For performance reasons there are different kind of passes:

\begin{block}{LLVM Passes}

% llvm-passes: hierarchy of LLVM passes.

\begin{tikzpicture}
[
  every node/.style={
    font=\scriptsize
  },
  extends/.style={
    draw,
    open triangle 90-
  },
  level distance=10mm,
  level 1/.style={
    sibling distance=20mm
  },
  level 3/.style={
    sibling distance=20mm
  }
]

\node {Pass}
  [edge from parent fork down,
   edge from parent/.style={extends}]

  child { node {CallGraphSCCPass}}
  child { node {ModulePass}
    child { node {ImmutablePass}}
  }
  child { node {FunctionPass}}
  child {node {LoopPass}}
  child {node {BasicBlockPass}};
\end{tikzpicture}

\centering
\end{block}
\end{frame}

\begin{frame}{LLVM Passes}
Each pass kind visits particular elements of a module:

\begin{description}[align=left, labelwidth=1cm]
\item[ImmutablePass] compiler configuration -- never run
\item[CallGraphSCCPass] post-order visit of CallGraph SCCs
\item[ModulePass] visit the whole module
\item[FunctionPass] visit functions
\item[LoopPass] post-order visit of loop nests
%\item[BasicBlockPass] visit basic blocks % DEPRECATED AND REMOVED
\item[RegionPass] visit a custom-defined region of code
\end{description}

\vfill
Specializations comes with restrictions:

\begin{itemize}
\item e.g. a \alert{FunctionPass} cannot add or delete functions
\item refer to ``Writing a LLVM Pass''~\cite{LOCAL:www/llvmWritingAPass}
      for accurate description of features and limitations of each kind of pass
\end{itemize}
\end{frame}

% \begin{frame}{Examples}
% Now we will see very simple passes:
%
% \begin{itemize}
% \item some of them are meaningless
% \item goal is to show you the LLVM API
% \end{itemize}
%
% \vfill
%
% The passes are:
% \begin{description}
% \item[instruction-count] simple instruction counting analysis
% \item[hello-llvm] optimization pass building an hello-world program
% \item[function-eraser] optimization pass removing ``small'' functions
% \end{description}
%
% \vfill
% Hint: take the LLVM pass writing tutorial~\cite{LOCAL:www/llvmWritingAPass}
% \end{frame}

\begin{frame}{What is Available Inside LLVM?}
LLVM provides passes performing basic transformations:

\begin{itemize}
\item variables promotion
\item loops canonicalization
\item \ldots
\end{itemize}

\vfill
They can be used to \alert{normalize/canonicalize} the input

\begin{itemize}
\item transform into a form analyzable for further passes
\item it is essential because keeps passes implementation manageable

\end{itemize}
\end{frame}

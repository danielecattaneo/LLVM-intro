% !TEX root = 01.tex

\section{Inside LLVM}


\begin{frame}{Using LLVM}
\begin{center}
LLVM is a \alert{compiler construction framework}\\
It operates on the \alert{LLVM-IR} language.\\
\bigskip
$\Downarrow$\\
\bigskip
Using LLVM \emph{by itself} does not make much sense!\\
Writing LLVM-IR by hand is unfeasible.
\end{center}
\end{frame}


\begin{frame}{Terminology}{Speaking About LLVM IR}
\begin{center}
LLVM-IR comes in 3 different flavours:
\end{center}
\bigskip
\begin{description}
\item[assembly] on-disk human-readable format\\(file extension: \texttt{.ll})
\item[bitcode] on-disk machine-oriented binary format\\(file extension: \texttt{.bc})
\item[in-memory] in-memory binary format\\(used during compilation process)
\end{description}
\bigskip
\begin{center}
All formats have the same expressiveness!
\end{center}
\vfill
\end{frame}


\begin{frame}{Frontends and Drivers}{C Language Family Front-end}
The LLVM-IR input to LLVM is provided by \alert{frontends}.
\begin{example}
\alert{Clang}\cite{LOCAL:www/clang} is the frontend for the C language family
\smallskip % why does \cite screw up the f*ing height of my block!?
\end{example}

The \alert{compiler driver} is the program that:
\begin{itemize}
\item Provides the interface to the user
\item Performs setup of the front end and LLVM itself. 
\end{itemize}

\begin{example}
The driver of \emph{Clang} is the \texttt{clang} executable (compatible with GCC)
\end{example}
\end{frame}


\begin{frame}{Frontends and Drivers}{Using the driver to produce LLVM-IR}
\begin{center}
\vfill
We can generate LLVM-IR assembly using the \texttt{clang} driver:\\
\bigskip
\texttt{clang -emit-llvm -S -o out.ll in.c}\\
\medskip
{\footnotesize If you want to generate bitcode instead:\\
\texttt{clang -emit-llvm -o out.bc in.c}\\
}
\bigskip
The compiler driver can also generate native code starting from 
LLVM-IR assembly\\
\smallskip
{\small(Like compiling an assembly file with GCC)}
\vfill
\end{center}
\end{frame}


\begin{frame}{Tools}{Playing with LLVM Passes}
Run one or more passes on the LLVM-IR on-demand by using \texttt{opt}:

\begin{itemize}
\item Syntax is like \texttt{clang} (supports even \texttt{-O1}, \texttt{-O2}...)
\item One command line argument per pass to run
\item Order of execution is the same as the argument order
\begin{itemize}
\item Different order, different results! (\alert{phase/stage ordering})
\end{itemize}
\end{itemize}

\vfill
Some useful passes for debugging (they do not transform anything):

{\small
\begin{description}[print dominator tree]
\item[print CFG] \texttt{opt -view-cfg input.ll}
\item[print dominator tree] \texttt{opt -view-dom input.ll}
\item[print current IR] \texttt{opt -print-module input.ll}
\end{description}
}

\vfill
\begin{example}
\begin{itemize}
\item Run \emph{mem2reg}, then view the CFG:
\begin{itemize}
\item \texttt{opt -mem2reg -view-cfg input.ll}
\end{itemize}
\end{itemize}
\end{example}
\end{frame}


\begin{frame}{Pass Hierarchy}
LLVM provides a lot of passes...

\begin{itemize}
\item Try \texttt{opt -help}!
\end{itemize}

\vfill
For performance reasons there are different kind of passes:

\begin{block}{LLVM Passes}

% llvm-passes: hierarchy of LLVM passes.

\begin{tikzpicture}
[
  every node/.style={
    font=\scriptsize
  },
  extends/.style={
    draw,
    open triangle 90-
  },
  level distance=10mm,
  level 1/.style={
    sibling distance=20mm
  },
  level 3/.style={
    sibling distance=20mm
  }
]

\node {Pass}
  [edge from parent fork down,
   edge from parent/.style={extends}]

  child { node {CallGraphSCCPass}}
  child { node {ModulePass}
    child { node {ImmutablePass}}
  }
  child { node {FunctionPass}}
  child {node {LoopPass}}
  child {node {BasicBlockPass}};
\end{tikzpicture}

\centering
\end{block}
\end{frame}


\begin{frame}{LLVM Passes}
Each kind of pass visits particular elements of a module:

\begin{description}[align=left, labelwidth=1cm]
\item[ImmutablePass] compiler configuration -- never run
\item[CallGraphSCCPass] post-order visit of CallGraph SCCs
\item[ModulePass] visit the whole module
\item[FunctionPass] visit functions
\item[LoopPass] post-order visit of loop nests
%\item[BasicBlockPass] visit basic blocks % DEPRECATED AND REMOVED
\item[RegionPass] visit a custom-defined region of code
\end{description}

\vfill
Specializations come with restrictions:

\begin{itemize}
\item e.g. a \alert{FunctionPass} cannot add or delete functions
\item refer to ``Writing a LLVM Pass''~\cite{LOCAL:www/llvmWritingAPass}
      for documentation on features and limitations of each kind of pass
\end{itemize}
\end{frame}


% \begin{frame}{Examples}
% Now we will see very simple passes:
%
% \begin{itemize}
% \item some of them are meaningless
% \item goal is to show you the LLVM API
% \end{itemize}
%
% \vfill
%
% The passes are:
% \begin{description}
% \item[instruction-count] simple instruction counting analysis
% \item[hello-llvm] optimization pass building an hello-world program
% \item[function-eraser] optimization pass removing ``small'' functions
% \end{description}
%
% \vfill
% Hint: take the LLVM pass writing tutorial~\cite{LOCAL:www/llvmWritingAPass}
% \end{frame}


\begin{frame}{Normalization passes in LLVM}
LLVM provides several \alert{normalization/canonicalization} passes:
\bigskip
\begin{itemize}
\item Variable-to-register promotion (\texttt{Mem2Reg})
\item Loop canonicalization (\texttt{LoopSimplify})
\item CFG canonicalization \& simplification (\texttt{SimplifyCFG})
\item \ldots
\end{itemize}
\bigskip
They are useful to make your life easier!
\end{frame}
